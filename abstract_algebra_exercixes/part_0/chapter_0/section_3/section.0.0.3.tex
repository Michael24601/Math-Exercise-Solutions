
\documentclass[12pt]{article}

\usepackage[margin=1in]{geometry}

% For using float option H that places figures exatcly where we want them
\usepackage{float}
% makes figure font bold
\usepackage{caption}
\captionsetup[figure]{labelfont=bf}
% For text generation
\usepackage{lipsum}
% For drawing
\usepackage{tikz}
% For smaller or equal sign and not divide sign
\usepackage{amssymb}
% For the diagonal fraction
\usepackage{xfrac}
% For enumerating exercise parts with letters instead of numbers
\usepackage{enumitem}
% For dfrac, which forces the fraction to be in display mode (large) e
% even in math mode (small)
\usepackage{amsmath}
% For degree sign
\usepackage{gensymb}
% For "\mathbb" macro
\usepackage{amsfonts}
\newcommand{\N}{\mathbb{N}}
\newcommand{\Z}{\mathbb{Z}}
\newcommand{\Q}{\mathbb{Q}}
\newcommand{\R}{\mathbb{R}}
\newcommand{\C}{\mathbb{C}}

% overline short italic
\newcommand{\olsi}[1]{\,\overline{{#1}}}
\newcommand{\lcm}{\text{lcm}}

\usetikzlibrary{shapes,positioning,fit,calc}

\usepackage{changepage} % for adjustwidth environment

\usepackage{listings}
\usepackage{xcolor}
\definecolor{mygreen}{rgb}{0,0.6,0}
\definecolor{mygray}{rgb}{0.5,0.5,0.5}
\definecolor{mymauve}{rgb}{0.58,0,0.82}
\lstset{
    language=Python,
    basicstyle=\ttfamily\footnotesize,
    numbers=left,
    numberstyle=\tiny\color{mygray},
    stepnumber=1,
    numbersep=5pt,
    backgroundcolor=\color{white},
    showspaces=false,
    showstringspaces=false,
    showtabs=false,
    frame=single,
    rulecolor=\color{black},
    tabsize=4,
    captionpos=b,
    breaklines=true,
    breakatwhitespace=false,
    keywordstyle=\color{blue},
    commentstyle=\color{mygreen},
    stringstyle=\color{mymauve},
    escapeinside={\%*}{*)},
    morekeywords={*,...}
}


\title{%
    \Huge Abstract Algebra \\
    \large by \\
    \Large Dummit and Foote \\~\\
    \huge Part 0: Preliminaries \\
    \LARGE Chapter 0: Preliminaries \\
    \Large Section 3: The Integers Modulo n
}
\date{2024-03-31}
\author{Michael Saba}

\begin{document}
    \pagenumbering{gobble}
    \maketitle
    \newpage
    \pagenumbering{arabic}

    \section*{Exercise 1}

    The elements of the residue classes of $\Z/18\Z$ are: \\
    The class $\olsi{0}$ is the set $\{ 18n \mid n \in \Z \}$. \\
    The class $\olsi{1}$ is the set $\{ 18n + 1 \mid n \in \Z \}$. \\
    The class $\olsi{2}$ is the set $\{ 18n + 2 \mid n \in \Z \}$. \\
    $\vdots$ \\
    The class $\olsi{17}$ is the set $\{ 18n + 17 \mid n \in \Z \}$. \\

    \section*{Exercise 2}
    Proof that the equivalence classes in $\Z/n\Z$
    are precisely $\olsi{1}$, $\olsi{2}\dots$ $\olsi{n-1}$: \\
    We know by the division algorithm,
    that for any integer $a$, 
    there exists two integers $q$ and $r$
    such that $a = nq + r$ where $0 \leqslant r < n$. \\
    We also know that $a \sim b$ if and only if
    they have the same remainder when divided by $n$.
    Since $r \in \{ 0, 1 \dots n-1 \}$,
    and since $0$, $1\dots$ $n-1$ are themselves the remainders
    when divided by $n$,
    and any member of an equivalence class can be its representative,
    it is clear that the equivalence classes of $\Z/n\Z$
    are $\olsi{0}$, $\olsi{1}\dots$ $\olsi{n-1}$. \\

    \section*{Exercise 3}
    Proof that if $a = a_n10^n + a_{n-1}10^{n-1} + \dots a_110 + a_0$
    is a positive integer,
    then $a \equiv (a_n + a_{n-1} + \dots a_1 + a_0) \mod 9$: \\
    We know from the book that $\olsi{a_1} = \olsi{b_1}$
    and $\olsi{a_2} = \olsi{b_2}$
    implies that $\olsi{a_1+a_2} = \olsi{b_1 + b_2}$
    and that $\olsi{a_1a_2} = \olsi{b_1b_2}$,
    meaning the sum and product operations are well defined
    for the classes.
    We can use modulo arithmetic to prove our statement. \\
    We want to show that, in $\Z/9\Z$
    \[  \olsi{a_n10^n + a_{n-1}10^{n-1} + \dots a_110 + a_0}
    = \olsi{a_n + a_{n-1} + \dots a_1 + a_0} \]
    We already know that $\olsi{10} = \olsi{1}$,
    and that $\olsi{10^k} = \olsi{10}^k = \olsi{1}^k = \olsi{1}$.
    Knowing that, we have
    \[  \olsi{a_n10^n + a_{n-1}10^{n-1} + \dots a_110 + a_0} \]
    \[ = \olsi{a_n10^n} +  \olsi{a_{n-1}10^{n-1}}
    + \dots  \olsi{a_110} +  \olsi{a_0} \]
    \[ = \olsi{a_n}\olsi{10^n} +  \olsi{a_{n-1}}\olsi{10^{n-1}}
    + \dots  \olsi{a_1}\olsi{10} +  \olsi{a_0} \]
    \[ = \olsi{a_n}\olsi{1} +  \olsi{a_{n-1}}\olsi{1}
    + \dots  \olsi{a_1}\olsi{1} +  \olsi{a_0} \]
    \[ = \olsi{a_n}\olsi{1} +  \olsi{a_{n-1}}\olsi{1}
    + \dots  \olsi{a_1}\olsi{1} +  \olsi{a_0} \]
    \[ = \olsi{a_n} +  \olsi{a_{n-1}}
    + \dots  \olsi{a_1} +  \olsi{a_0} \]
    which proves our statement. \\

    \section*{Exercise 4 $***$}
    The remainder of $37^{100}$ when divided by $29$
    can be calculated through arithmetic in $\Z/29\Z$,
    where we know addition and multiplication is well defined. \\
    We know that
    \[ \olsi{37} = \olsi{8} \]
    \[ \olsi{37^2} = \olsi{37}^2 = \olsi{8}^2
    = \olsi{8^2} = \olsi{64} = \olsi{6} \]
    \[ \olsi{37^4} = \olsi{37^2}^2 = \olsi{6}^2
    = \olsi{6^2} = \olsi{36} = \olsi{7} \]
    \[ \olsi{37^8} = \olsi{37^4}^2 = \olsi{7}^2
    = \olsi{7^2} = \olsi{49} = \olsi{20} \]
    \[ \olsi{37^{16}} = \olsi{37^8}^2 = \olsi{20}^2
    = \olsi{20^2} = \olsi{400} = \olsi{23} \]
    \[ \olsi{37^{32}} = \olsi{37^{16}}^2 = \olsi{23}^2
    = \olsi{23^2} = \olsi{529} = \olsi{7} \]
    \[ \olsi{37^{64}} = \olsi{37^{32}}^2 = \olsi{7}^2
    = \olsi{7^2} = \olsi{49} = \olsi{20} \]
    We can thus write
    \[ \olsi{37^{100}} = \olsi{37^{64 + 32 + 4}}
    = \olsi{37^{64}37^{32}37^{4}}
    = \olsi{37^{64}}\olsi{37^{32}}\olsi{37^{4}} 
    = \olsi{20}\olsi{7}\olsi{7}
    = \olsi{20 \cdot 7 \cdot 7}
    = \olsi{2401} = \olsi{23} \]
    Since $23$ is the smallest positive integer in the equivalence
    class of $37^{100}$
    (the only positive integer smaller than $29$),
    we conclude that it is the remainder of $37^{100}$
    when divided by $29$. \\

    \section*{Exercise 5}
    To compute the last two digits of $9^{1500}$
    is equivalence to computing $9^{1500} \mod 100$.
    Using the same method as exercise 0.0.3.4,
    we find that $9^{1500} \equiv 1 \mod 100$.
    Since $1$ is the smallest positive integer
    in the equivalence class,
    it is the remainder of $9^{1500}$ when divided by $100$,
    which means that $9^{1500}$ ends with $01$. \\

    \section*{Exercise 6}
    Proof that the squares of all elements in $\Z/4\Z$
    are just $\olsi{0}$ and $\olsi{1}$: \\
    We know that addition and multiplication are well defined
    in $\Z/n\Z$.
    We also know that
    \[\Z/n\Z = \{ \olsi{0}, \olsi{1}, \olsi{2}, \olsi{3} \}\]
    So the squares of all the elements are
    \[ \olsi{0}^2 = \olsi{0^2} = \olsi{0} \]
    \[ \olsi{1}^2 = \olsi{1^2} = \olsi{1} \]
    \[ \olsi{2}^2 = \olsi{2^2} = \olsi{4} = \olsi{0} \]
    \[ \olsi{3}^2 = \olsi{3^2} = \olsi{9} = \olsi{1} \]

    \section*{Exercise 7}
    Proof that if $a$ and $b$ are any integers,
    then $a^2 + b^2$ never leaves a remainder of $3$
    when divided by $4$: \\
    Multiplication and addition are well defined in $\Z/n\Z$,
    So all we need to show is that
    \[ \olsi{a^2 + b^2} \neq \olsi{3} \]
    which means that 
    \[ \olsi{a^2} + \olsi{b^2} \neq \olsi{3} \]
    \[ \olsi{a}^2 + \olsi{b}^2 \neq \olsi{3} \]
    From exercise 0.0.3.6,
    we know that the squares of all elements in $\Z/4\Z$
    of the form $\olsi{a}^2$
    are either $\olsi{0}$ or $\olsi{1}$.
    So we only need to show that the above equation
    holds when $a$ and $b$ are equal to these two values:
    \[ \olsi{0} + \olsi{0} = \olsi{0 + 0} = \olsi{0} \]
    \[ \olsi{1} + \olsi{1} = \olsi{1 + 1} = \olsi{2} \]
    \[ \olsi{0} + \olsi{1} = \olsi{1 + 0} = \olsi{1} \]
    \[ \olsi{1} + \olsi{0} = \olsi{0 + 1} = \olsi{1} \]
    Since the answer is never $\olsi{3}$,
    the remainder of $a^2 + b^2$ when divided by $4$
    is never $3$. \\

    \section*{Exercise 8 $***$}
    Proof that for any non-zero integers $a, b, c \in \Z - \{ 0 \}$, 
    the equation $a^2 + b^2 = 3c^2$ has no solutions: \\
    We can use a proof by smallest counterexample,
    in which we assume we have a fully reduced solution,
    and show that we can always derive a more reduced solution,
    which contradicts the Well Ordering Principle
    of the positive integers. \\
    First we need to show that each of $a, b, c$
    are divisible by $2$.
    Consider the equation in $\Z/4\Z$,
    which we can do since addition and multiplication are well defined
    in $\Z/n\Z$.
    For the equation to hold,
    we would need to have $a^2 + b^2 \equiv 3c^2 \mod 4$. \\
    We know from exercise 0.0.3.7
    that $\olsi{a^2 + b^2} \neq \olsi{3}$
    for any $a, b \in \Z$. \\
    We know from exercise 0.0.3.6 that $\olsi{c^2} = \olsi{0}$
    or $\olsi{c^2} = \olsi{1}$ for any $c \in \Z$,
    which means that
    \[ \olsi{3c^2} = \olsi{3} \cdot \olsi{1}
    = \olsi{3 \cdot 1} = \olsi{3} 
    \quad \text { or } \quad
    \olsi{3c^2} = \olsi{3} \cdot \olsi{0}
    = \olsi{3 \cdot 0} = \olsi{0} \]
    So, in order to be equal, the two sides of the equation must
    both be equal to $\olsi{0}$. \\
    The only way to have $\olsi{3c^2} = \olsi{0}$,
    if to have $\olsi{c} = \olsi{0}$.
    We know from exercise 0.0.3.6 that this happens if
    $\olsi{c} = \olsi{0}$ or if $\olsi{c} = \olsi{2}$.
    If $c \equiv 0 \mod 4$,
    then $4 \mid c$, which means that $c = 4p = 2(2p)$
    for some $p \in \Z$.
    If $c \equiv 2 \mod 4$,
    then $4 \mid c - 2$, which means that $c - 2 = 4q = 4q + 2 = 2(2q + 1)$
    for some $q \in \Z$.
    Either way, we now know that $2 \mid c$. \\
    The only way to have $\olsi{a^2 + b^2} = \olsi{0}$,
    according to exercise 0.0.3.7,
    is to have $\olsi{a}$ and $\olsi{b}$ be equal to $\olsi{0}$.
    This means that both $a$ and $b$ are divisible by $4$,
    and by extension, by $2$. \\
    This means that $a, b, c$ are all divisible by $2$.
    If we write
    \[ a = 2m \qquad b = 2r \qquad c = 2s \]
    we can show that
    \[ a^2 = 4m^2 \qquad b = 4r^2 \qquad c^2 = 4s^2 \]
    which means that $a^2, b^2, c^2$ all divisible by $4$. \\
    We now assume that $a^2 + b^2 = 3c^2$
    is the most reduced it can be
    (e.g. that we can't simplify it further).
    However, we know that if $a^2 + b^2 = 3c^2$,
    then $4m^2 + 4r^2 = 3 \cdot 4s^2$,
    which means that $m^2 + r^2 = 3s^2$
    is a more reduced form of the same equation. 
    Since we can repeat the above argument indefinitely,
    where $m^2, r^2, s^2$ are also divisible by $4$,
    we conclude that no smallest numbers $a^2$, $b^2$ and $c^2$
    can exist to satisfy the equation
    by the Well Ordering of $\Z^+$
    (since $a^2, b^2, c^2 \in \Z^+$,
    and none are zero,
    so dividing them by $4$ always produces smaller numbers). \\

    \section*{Exercise 9 $***$}
    Proof that the square of any odd integer leaves
    a remainder of $1$ when divided by $8$: \\
    To show that this is the case, since $m$ is odd,
    we can write $m = 2n+1$ for some integer $n$.
    Then
    \[ m^2 = (2n+1)^2 = 4m^2+4m+1 = 4(m^2 + m) + 1
    = 4(m)(m+1) + 1 \].
    Since $m$ and $m+1$ have opposite parities,
    one of them must be even.
    Let's say that $a$ is the even term and $b$ is the odd term.
    Then we can write $a = 2c$ for some $c \in \Z$,
    which means that $m^2 = 4ab + 1 = 4(2c)b + 1 = 8cb + 1$.
    So $m^2 - 1 = 8cb$ where $cb \in \Z$,
    which tells us that $8 \mid m^2 - 1$.
    This means that $m^2 \equiv 1 \mod 8$,
    completing the proof. \\

    \section*{Exercise 10 $***$}
    Proof that the number of elements of $(\Z/n\Z)^\times$
    is $\varphi(n)$ where $\varphi$ is Euler's Phi function: \\
    We know that $(\Z/n\Z)^\times$ is the set of elements of $\Z/n\Z$
    that have a multiplicative inverse.
    First we need to show that 
    \[ A = \{ \olsi{a} \mid \olsi{a} \in \Z/n\Z
    \text{ and } \gcd(a, n) = 1 \} 
    \text{ is the same as } (\Z/n\Z)^\times \]
    This means that we need to show that,
    for any equivalence class $\olsi{a}$,
    $\olsi{a}$ has a multiplicative inverse in $\Z/n\Z$
    if and only if $\gcd(a, n) = 1$.
    Another way of saying that is that we need to show
    that $(\Z/n\Z)^\times \subseteq A$
    and that $A \subseteq (\Z/n\Z)^\times$,
    which would imply that $A = (\Z/n\Z)^\times$. \\
    Note that because we are looking for the inverse of a set,
    we need to show that the operation is well defined
    (we need to prove it for any representative $a$ of $\olsi{a}$). \\
    First we note that if $\gcd(a, n) = 1$,
    then by the Euclidian Algorithm, we can find integers $x, y \in \Z$
    such that $ax + ny = 1$,
    which means that $ax - 1 = ny$,
    which in turn implies that $n \mid ax - 1$.
    This tells us that $ax \equiv 1 \mod n$,
    which makes $x$ the multiplicative inverse of $a$.
    Note that if $b \in \olsi{a}$,
    then $b \equiv a \mod n$,
    so $xb \equiv a \mod n$,
    which means that if $a$ has a multiplicative inverse $x$,
    then so does $b$. \\
    So if $\gcd(a, n) = 1$
    we know that $\olsi{a} \in (\Z/n\Z)^\times$,
    which means that $(\Z/n\Z)^\times \subseteq A$. \\
    Next, we note that if $\olsi{a}$ has a multiplicative inverse
    $x$ in $\Z/n\Z$,
    then $ax \equiv 1 \mod n$.
    So $n \mid ax - 1$,
    which means that $ax - 1 = ny$ for some $y \in \Z$.
    We can rearrange the equation to show that $ax - ny = 1$.
    If $\gcd(a, n) = d > 1$,
    then we could take $d$ as a factor in the left hand side,
    which gives us $dc = 1$ where $c = \dfrac{ax - ny}{\gcd(a, n)}$
    is an integer.
    This is impossible,
    since $d$ is an integer not equal $1$,
    and $c$ is also an integer,
    which means that their product can't be equal to $1$.
    Thus we can conclude that $\gcd(a, n) = 1$.
    Note that if $b \in \olsi{a}$,
    then $b$ has the same multiplicative inverse $x$,
    since $bx \equiv ax \mod n$,
    so if $\gcd(a, n) = 1$, then $\gcd(b, n) = 1$ too.
    So if $\olsi{a} \in (\Z/n\Z)^\times$,
    then all elements $a \in \olsi{a}$
    are relatively prime to $n$,
    which means that $A \subseteq (\Z/n\Z)^\times$. \\
    We can thus conclude that $A = (\Z/n\Z)^\times$. \\
    We just showed that $A = (\Z/n\Z)^\times$,
    where we also had to show that it applies
    no matter which representative of $a$ we pick,
    which makes it well definded.
    We can thus choose the representatives
    $\{1, 2 \dots n \}$
    to represent
    the $n$ classes $\olsi{0}, \olsi{1}, \olsi{2} \dots \olsi{n-1}$.
    This means that if any of $\{1, 2 \dots n \}$
    are relatively prime to $n$,
    their respective class has to be in $(\Z/n\Z)^\times$.
    By definition,
    this means that there will be $\varphi(n)$
    residue classes in $(\Z/n\Z)^\times$,
    since of the integers $\{1, 2 \dots n \}$,
    $\varphi(n)$ are relatively prime to $n$ by definition. \\

    \subsection*{Exercise 11}
    Proof that if $\olsi{a}, \olsi{b} \in (\Z/n\Z)^\times$,
    then $\olsi{a} \cdot \olsi{b} \in (\Z/n\Z)^\times$: \\
    We already know that multiplication
    and multiplicative inverses are well defined in $\Z/n\Z$
    (according to the chapter and to exercise 0.0.3.10),
    so we don't need to show that they are. \\
    If $\olsi{a}, \olsi{b} \in (\Z/n\Z)^\times$,
    then they have inverses $\olsi{x}$ and $\olsi{y}$
    such that $\olsi{a} \cdot \olsi{x} = \olsi{b} \cdot \olsi{y} = 1$.
    Now consider the class $\olsi{y} \cdot \olsi{x}$
    (which is equal to $\olsi{yx}$). \\
    From the well definition of multiplication we know that
    \[ (\olsi{a} \cdot \olsi{b}) \cdot (\olsi{y} \cdot \olsi{x})
    = \olsi{ab} \cdot \olsi{yx}
    = \olsi{abyx}
    = \olsi{a} \cdot \olsi{bx} \cdot \olsi{x}
    = \olsi{a} \cdot (\olsi{b} \cdot \olsi{x}) \cdot \olsi{x} \]
    \[ = \olsi{a} \cdot \olsi{1} \cdot \olsi{x}
    = \olsi{a \cdot 1} \cdot \olsi{x}
    = \olsi{a} \cdot \olsi{x}
    = \olsi{1} \]
    So $\olsi{y} \cdot \olsi{x}$
    is the multiplicative inverse of $\olsi{a} \cdot \olsi{b}$.
    This means that $\olsi{a} \cdot \olsi{b} \in (\Z/n\Z)^\times$
    since it has a multiplicative inverse. \\

    \subsection*{Exercise 12}
    Proof that, for $n, a \in \Z$ such that $n > 1$
    and $1 \leqslant a \leqslant n$,
    if $a$ and $n$ are not relatively prime,
    then there exists an integer $b$ such that
    $1 \leqslant b < n$ and $ab \equiv 0 \mod n$,
    which means that there there cannot exist an integer $c$
    such that $ac \equiv 1 \mod n$: \\
    If $\gcd(a, n) = d > 1$,
    then we can take $b$ to be $\dfrac{n}{d}$.
    We can clearly see that $1 \leqslant b < n$
    since $d > 1$.
    We can also see that $\dfrac{a}{d}$ is an integer $p \in \Z$
    since $d \mid a$,
    which means that $ab = a\dfrac{n}{d} = pn$,
    which in turn tells us that $n \mid ab$.
    Thus $ab \equiv 0 \mod n$. \\
    However, assume now that there exists an integer $c \in \Z$
    such that $ac \equiv 1 \mod n$.
    We know by definition that if $x \equiv y \mod n$,
    then $xz \equiv y \mod n$ for any $x, y, z \in \Z$. 
    So $abc \equiv 0 \mod n$
    since $ab \equiv 0 \mod n$.
    But this is not possible;
    for $abc$ to have remainder $0$ when divided by $n$,
    it needs to have a factor of $n$ in it;
    we know that $ac = nq + 1$ for some integer $q \in \Z$,
    and we know that $1 \leqslant b < n$,
    so multiplying $ac$ by $b$
    produces $nqb + b$ where $b$ does not contain a factor of $n$,
    so $abc \equiv 0 \mod n$ is a contradiction.
    We thus conclude that $c$ can't exist. \\

    \subsection*{Exercise 13}
    Proof that, for $n, a \in \Z$ such that $n > 1$
    and $1 \leqslant a \leqslant n$,
    if $a$ and $n$ are relatively prime,
    then there exists an $c \in \Z$
    such that $ac \equiv 1 \mod n$: \\
    If $\gcd(a, n) = 1$,
    then by the Euclidian Algorithm, we can find integers $x, y \in \Z$
    such that $ax + ny = 1$,
    which means that $ax - 1 = ny$,
    which in turn implies that $n \mid ax - 1$.
    This tells us that $ax \equiv 1 \mod n$.
    We then just take $c = x$. \\

    \subsection*{Exercise 14}
    We already proved in exercise 0.0.3.10 that
    \[ (\Z/n\Z)^\times = \{ \olsi{a} \mid \olsi{a} \in \Z/n\Z
    \text{ and } \gcd(a, n) = 1 \} \]
    but now we need to conclude it
    from exercises 0.0.3.12 and 0.0.3.13: \\
    We showed in exercises 0.0.3.12 and 0.0.3.13
    that if $\gcd(a, n) = 1$,
    then there exists an integer $c$ such that $ac \equiv 1 \mod n$,
    and that if $\gcd(a, n) > 1$,
    then there does not exist
    an integer $c$ such that $ac \equiv 1 \mod n$.
    Contrapositively,
    the second statement states that
    if there exists an integer $c$ such that $ac \equiv 1 \mod n$,
    then $\gcd(a, n) = 1$.
    Combined, the statements claim that
    $\gcd(a, n) = 1$ if and only if an integer $c$ exists
    such that $ac \equiv 1 \mod n$. \\
    We thus conclude that $\olsi{a}$
    has a multiplicative inverse $\olsi{c}$ in $(\Z/n\Z)^\times$ 
    if and only if $\gcd(a, n) = 1$
    (for any representative $a$ of $A$). \\
    Note that this statement is until now, only true for 
    integers $a$ such that $1 \leqslant a \leqslant n$;
    it is not yet well defined since we did not prove it
    for all representatives of $\olsi{a}$;
    however, we were not asked to do that in this problem.
    For the full proof, see $0.0.3.14$. \\
    Now, we need to verify this for $n = 12$:
    We know that only $1, 5, 7, 11$ are relatively prime to $12$
    (out of the integers between $1$ and $12$).
    If we draw the mutltiplication table of all numbers
    between $1$ and $12$ with each other (modulo $12$),
    we would see that $5 \cdot 5 = 25 \equiv 1 \mod 12$,
    $7 \cdot 7 = 49 \equiv 1 \mod 12$,
    and $11 \cdot 11 = 121 \equiv 1 \mod 12$
    are the only combinations that produce $1$:

    \begin{figure}[H]
        \centering
        \begin{tabular}{|c|c|c|c|c|c|c|c|c|c|c|c|c|}
        \hline
        $\times$ & 1 & 2 & 3 & 4 & 5 & 6 & 7 & 8 & 9 & 10 & 11 & 12 \\
        \hline
        1 & 1 & 2 & 3 & 4 & 5 & 6 & 7 & 8 & 9 & 10 & 11 & 0 \\
        2 & 2 & 4 & 6 & 8 & 10 & 0 & 2 & 4 & 6 & 8 & 10 & 0 \\
        3 & 3 & 6 & 9 & 0 & 3 & 6 & 9 & 0 & 3 & 6 & 9 & 0 \\
        4 & 4 & 8 & 0 & 4 & 8 & 0 & 4 & 8 & 0 & 4 & 8 & 0 \\
        5 & 5 & 10 & 3 & 8 & \underline{1} & 6 & 11 & 4 & 9 & 2 & 7 & 0 \\
        6 & 6 & 0 & 6 & 0 & 6 & 0 & 6 & 0 & 6 & 0 & 6 & 0 \\
        7 & 7 & 2 & 9 & 4 & 11 & 6 & \underline{1} & 8 & 3 & 10 & 5 & 0 \\
        8 & 8 & 4 & 0 & 8 & 4 & 0 & 8 & 4 & 0 & 8 & 4 & 0 \\
        9 & 9 & 6 & 3 & 0 & 9 & 6 & 3 & 0 & 9 & 6 & 3 & 0 \\
        10 & 10 & 8 & 6 & 4 & 2 & 0 & 10 & 8 & 6 & 4 & 2 & 0 \\
        11 & 11 & 10 & 9 & 8 & 7 & 6 & 5 & 4 & 3 & 2 & \underline{1} & 0 \\
        12 & 0 & 0 & 0 & 0 & 0 & 0 & 0 & 0 & 0 & 0 & 0 & 0 \\
        \hline
        \end{tabular}
        \caption{The multiplication table modulo $12$.}
        \label{fig:figure1}
    \end{figure}

    \subsection*{Exercise 15}
    For each of these pairs of numbers $a$ and $n$,
    we need to show that $a$ and $n$ are relatively prime,
    and we need to find the multiplicative inverse of $\olsi{a}$
    in $\Z/n\Z$: \\
    What we can do is use the Euclidian Algorithm
    to show that $\gcd(a, n) = 1$ for each case,
    then we can find the integers $x$ and $y$
    such that $ax + by = \gcd(a, n) = 1$ using substitution,
    which will tell us that $x$ is a representative of
    the inverse class of $\olsi{a}$. \\
    Or we can use the computer program we wrote in exercise $0.0.2.9$. \\ 
    \begin{enumerate}[label=\textbf{\alph*.}]
        \item 
            For $a = 13$ and $b = 20$,
            we will get $x = -3$ and $y = 2$.
            This means that $\olsi{-3} = \olsi{-17}$
            is the inverse of $\olsi{13}$.
        \item 
            For $a = 69$ and $b = 89$,
            we will get $x = 40$ and $y = -31$.
            This means that $\olsi{40}$
            is the inverse of $\olsi{69}$.
        \item 
            For $a = 1891$ and $b = 3797$,
            we will get $x = 253$ and $y = -126$.
            This means that $\olsi{253}$
            is the inverse of $\olsi{1891}$.
        \item 
            For $a = 6003722857$ and $b = 77695236973$,
            we will get $x = -220$ and $y = 17$.
            This means that $\olsi{-220} = \olsi{77695236753}$
            is the inverse of $\olsi{6003722857}$.
    \end{enumerate}
 
    \subsection*{Exercise 16}
    This is a \textit{Python} program that can compute
    additions and multiplications modulo $n$,
    and print the results as the least residue of
    the sums and products (the representative between $0$ and $n-1$).
    This program also prints the least residue representative
    of the inverse class of the input if the input is relatively
    prime to $n$ (the representative between $0$ and $n-1$):
    \begin{lstlisting}
        # Will complete later
    \end{lstlisting}


\end{document}
