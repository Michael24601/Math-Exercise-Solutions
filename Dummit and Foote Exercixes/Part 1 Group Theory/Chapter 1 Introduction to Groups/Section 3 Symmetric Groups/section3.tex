
\documentclass[12pt]{article}

\usepackage[margin=1in]{geometry}

\usepackage{float}
% makes figure font bold
\usepackage{caption}
\captionsetup[figure]{labelfont=bf}
% For text generation
\usepackage{lipsum}
% For drawing
\usepackage{tikz}
% For manipulating coordinates
\usetikzlibrary{calc}
% For smaller or equal sign and not divide sign
\usepackage{amssymb}
% For the diagonal fraction
\usepackage{xfrac}
% For enumerating exercise parts with letters instead of numbers
\usepackage{enumitem}
% For dfrac, which forces the fraction to be in display mode (large) e
% even in math mode (small)
\usepackage{amsmath}
% For degree sign
\usepackage{gensymb}
% For "\mathbb" macro
\usepackage{amsfonts}
\newcommand{\N}{\mathbb{N}}
\newcommand{\Z}{\mathbb{Z}}
\newcommand{\Q}{\mathbb{Q}}
\newcommand{\R}{\mathbb{R}}
\newcommand{\C}{\mathbb{C}}

% overline short italic
\newcommand{\olsi}[1]{\,\overline{\!{#1}}}

\title{%
    \Huge Abstract Algebra \\
    \large by \\
    \Large Dummit and Foote \\~\\
    \huge Part 1: Group Theory \\
    \LARGE Chapter 1: Introduction to Groups \\
    \Large Section 3: Symmetric Groups
}
\date{2023-07-14}
\author{Michael Saba}

\begin{document}
    \pagenumbering{gobble}
    \maketitle
    \newpage
    \pagenumbering{arabic}


    \section*{Exercise 1}
    $\sigma = (1\;3\;5)(2\;4)$ \\
    $\tau = (1\;5)(2\;4)$ \\
    $\sigma^2 = (1\;5\;3)$ \\
    $\sigma \tau = (2\;5\;4\;3)$ \\ 
    $\tau \sigma = (1\;2\;4\;3)$ \\
    $\tau^2 \sigma = (1\;3\;5)(2\;4)$ \\


    \section*{Exercise 2}
    $\sigma = (1\;13\;5\;10)(3\;15\;8)(4\;14\;11\;7\;12\;9)$ \\
    $\tau = (1\;14)(2\;9\;13\;15\;4)(3\;10)(5\;12\;7)(8\;11)$ \\
    $\sigma^2 = (1\;5)(12\;10)(3\;8\;15)(4\;11\;12)(14\;7\;9)$ \\
    $\sigma \tau = (1\;11\;3)(2\;4)(5\;9)(7\;10\;15\;14\;13\;8)$ \\ 
    $\tau \sigma = (1\;15\;11\;5\;3\;4)(2\;9)(8\;10\;14)(12\;13)$ \\
    $\tau^2 \sigma = (1\;4\;14\;11\;12\;15\;8\;3\;2\;13\;7\;5\;10)$ \\


    \section*{Exercise 3}
    In exercise 1.1.3.1: \\
    $|\sigma| = 6$,
    $|\tau| = 2$, 
    $|\sigma^2| = 3$, 
    $|\sigma \tau| = 4$, 
    $|\tau \sigma| = 4$,
    $|\tau^2 \sigma| = 6$. \\ 
    In exercise 1.1.3.2: \\
    $|\sigma| = 12$, 
    $|\tau| = 30$,
    $|\sigma^2| = 6$,
    $|\sigma \tau| = 6$,
    $|\tau \sigma| = 6$,
    $|\tau^2 \sigma| = 13$. 


    \section*{Exercise 4}
    \begin{enumerate}[label=\textbf{\alph*.}]
        \item 
            In $S_3$: \\
            $|1| = 1$,
            $|(1\;2)| = 2$,
            $|(1\;3)| = 2$,
            $|(2\;3)| = 2$,
            $|(1\;2\;3)| = 3$,
            $|(1\;3\;2)| = 3$.
        \item 
            In $S_4$: \\
            $|1| = 1$,
            $|(1\;2)| = 2$,
            $|(1\;3)| = 2$,
            $|(2\;3)| = 2$,
            $|(1\;4)| = 2$,
            $|(2\;4)| = 2$,
            $|(3\;4)| = 2$,
            $|(1\;2\;3)| = 3$,
            $|(1\;3\;2)| = 3$,
            $|(1\;2\;3)| = 3$,
            $|(1\;4\;3)| = 3$,
            $|(1\;4\;2)| = 3$,
            $|(1\;2\;4)| = 3$,
            $|(1\;3\;4)| = 3$,
            $|(2\;3\;4)| = 3$,
            $|(1\;2\;3\;4)| = 4$,
            $|(1\;2\;4\;3)| = 4$,
            $|(1\;3\;2\;4)| = 4$,
            $|(1\;3\;4\;2)| = 4$,
            $|(1\;4\;2\;3)| = 4$,
            $|(1\;4\;3\;2)| = 4$,
            $|(1\;3)(2\;4)| = 2$,
            $|(1\;2)(3\;4)| = 2$,
            $|(1\;4)(2\;3)| = 2$.
    \end{enumerate}   


    \section*{Exercise 5}
    $|(1\;12\;10\;8\;4)(2\;13)(5\;11\;7)(6\;9)| = lcm(5, 2, 3) = 30$


    \section*{Exercise 6}
    The elements of order 4 are: \\
    $(1\;2\;3\;4)$,
    $(1\;2\;4\;3)$,
    $(1\;3\;2\;4)$,
    $(1\;3\;4\;2)$,
    $(1\;4\;2\;3)$,
    $(1\;4\;3\;2)$.


    \section*{Exercise 7}
    The elements of order 2 are: \\
    $(1\;2)$,
    $(1\;3)$,
    $(2\;3)$,
    $(1\;4)$,
    $(2\;4)$,
    $(3\;4)$,
    $(1\;3)(2\;4)$,
    $(1\;2)(3\;4)$,
    $(1\;4)(2\;3)$.


    \section*{Exercise 8}
    Proof that if $\Omega = \{1, 2, 3 ...\}$,
    then $S_{\Omega}$ has an infinite order: \\
    Consider the permutations that swap the elements 1 and $n$
    for every $n \in \Omega$.
    Since we have infinite $n$s, the number of permutations must also be
    infinite.
      

    \section*{Exercise 9}
    \begin{enumerate}[label=\textbf{\alph*.}]
        \item 
            $\sigma = (1\;2\;3\;4\;5\;6\;7\;8\;9\;10\;11\;12)$, \\
            $\sigma^2 = (1\;3\;5\;7\;9\;11)(2\;4\;6\;8\;10\;12)$, \\
            $\sigma^3 = (1\;4\;7\;10)(2\;5\;8\;11)(3\;6\;9\;12)$, \\
            $\sigma^4 = (1\;5\;9)(2\;6\;10)(3\;7\;11)(4\;8\;12)$, \\
            $\sigma^5 = (1\;7)(2\;8)(3\;9)(4\;10)(5\;11)(6\;12)$, \\
            $\sigma^6 = (1\;6\;11\;4\;9\;2\;7\;12\;5\;10\;3\;8)$, \\
            $\sigma^7 = (1\;8\;3\;10\;5\;12\;7\;2\;9\;4\;11\;6)$, \\
            $\sigma^8 = (1\;9\;5)(2\;10\;6)(3\;11\;7)(4\;12\;8)$, \\
            $\sigma^9 = (1\;10\;7\;4)(2\;11\;8\;5)(3\;12\;9\;6)$, \\
            $\sigma^{10} = (1\;11\;9\;7\;5\;3)(2\;12\;10\;8\;6\;4)$, \\
            $\sigma^{11} = (1\;12\;11\;10\;9\;8\;7\;6\;5\;4\;3\;2)$, \\
            $\sigma^{12} = \sigma^0 = 1$. \\
            Since 12 is the smallest $n$ for which $\sigma^n = 1$, 
            $|\sigma| = 12$.
            So by the division algorithm, any integer $m = nq + r$
            where $0 \leqslant r < n$.
            So for $n = 12$, $\sigma^m = \sigma^{nq + r}
            = \sigma^{12q}\sigma^r
            = (\sigma^12)^q\sigma^r
            = 1^q\sigma^r 
            = \sigma^r$.
            So $\forall m \in \Z, \sigma^m = \sigma^{m \mod 12}$.
            So since $|\sigma^k|$ is a 12-cycle for $k = \{ 1, 6, 7, 11 \}$,
            $\forall i \in \Z$, 
            $\sigma^{12i + 1}$, $\sigma^{12i + 6}$, $\sigma^{12i + 7}$, 
            and $\sigma^{12i + 11}$ are all 12-cycles, .
        \item 
            $\tau = (1\;2\;3\;4\;5\;6\;7\;8)$, \\
            $\tau^2 = (1\;3\;5\;7)(2\;4\;6\;8)$, \\
            $\tau^3 = (1\;4\;7\;2\;5\;8\;3\;6)$, \\
            $\tau^4 = (1\;5)(2\;6)(3\;7)(4\;8)$, \\
            $\tau^5 = (1\;6\;3\;8\;5\;2\;7\;4)$, \\
            $\tau^6 = (1\;7\;5\;3)(2\;8\;6\;4)$, \\
            $\tau^7 = (1\;8\;7\;6\;5\;4\;3\;2)$, \\
            $\tau^8 = \tau^0 = 1$. \\
            Since $|\tau^k|$ is an 8-cycle for $k = \{ 1, 3, 5, 7 \}$,
            $\forall i \in \Z$, 
            $\tau^{8i + 1}$, $\tau^{8i + 3}$, $\tau^{8i + 5}$, 
            and $\tau^{8i + 7}$ are al 8-cycles.
        \item
            $\omega = (1\;2\;3\;4\;5\;6\;7\;8\;9\;10\;11\;12\;13\;14)$, \\
            $\omega^2 = (1\;3\;5\;7\;9\;11\;13)(2\;4\;6\;8\;10\;12\;14)$, \\
            $\omega^3 = (1\;4\;7\;10\;13\;2\;5\;8\;11\;14\;3\;6\;9\;12)$, \\
            $\omega^4 = (1\;5\;9\;13\;3\;7\;11)(2\;6\;10\;14\;4\;8\;12)$, \\
            $\omega^5 = (1\;6\;11\;2\;7\;12\;3\;8\;13\;4\;9\;14\;5\;10)$, \\
            $\omega^6 = (1\;7\;13\;5\;11\;3\;9)(2\;8\;14\;6\;12\;4\;10)$, \\
            $\omega^7 = (1\;8)(2\;9)(3\;10)(4\;11)(5\;12)(6\;13)(7\;14)$, \\
            $\omega^8 = (1\;9\;3\;11\;5\;13\;7)(2\;10\;4\;12\;6\;14\;8)$, \\
            $\omega^9 = (1\;10\;5\;14\;9\;4\;13\;8\;3\;12\;7\;2\;11\;6)$, \\
            $\omega^{10} = (1\;11\;7\;3\;13\;9\;5)(2\;12\;8\;4\;14\;10\;6)$, \\
            $\omega^{11} = (1\;12\;9\;6\;3\;14\;11\;8\;5\;2\;13\;10\;7\;4)$, \\
            $\omega^{12} = (1\;13\;11\;9\;7\;5\;3)(2\;14\;12\;10\;8\;6\;4)$, \\
            $\omega^{13} = (1\;14\;13\;12\;11\;10\;9\;8\;7\;6\;5\;4\;3\;2)$, \\
            $\omega^{14} = .\omega^0 = 1$. \\
            Since $|\omega^k|$ is a 14-cycle for
            $k = \{ 1, 3, 5, 7, 9, 11, 13 \}$, $\forall i \in \Z$, 
            $\omega^{14i + 1}$, $\omega^{14i + 3}$, $\omega^{14i + 5}$,
            $\omega^{14i + 7}$, $\omega^{14i + 9}$, $\omega^{14i + 11}$,
            and $\omega^{14i + 13}$ are 14-cycles.
    \end{enumerate}


    \section*{Exercise 10 $***$}
    Proof that in $\sigma = (a_1, a_2, ... a_m), \sigma^i(a_k) = a_q$
    where $q \equiv (k + i) \mod m$: \\
    $\sigma$ moves $a_k$ to $a_{k+1}$.
    $\sigma^2$ moves $a_k$ to $a_{k+1}$, then $a_{k+1}$ to $a_{k+2}$.
    So as long as $k + i \leqslant m$,
    then $\sigma^i$ moves $a_k$ to $a_{k+i}$,
    where, trivially, $(k + i) \equiv (k + i) \mod m$. \\
    However, if $k + i > m$,
    then we start by dividing $k + i$ into $b = (m - k)$
    and $c = i - b = i + k - m$.
    Here, $b$ is the largest to which $\sigma$ sends $a_k$ to a value 
    within the range of $a_k$ to $a_m$.
    In other words, $\sigma^b(a_k) = a_m$. \\
    So $\sigma^i(a_k) = \sigma^{b + c}(a_k)
    = \sigma^c(\sigma^b(a_k))
    = \sigma^c(a_m)$. \\
    At this point, if $i + k - m \leqslant m$,
    then $\sigma^c(a_m) = \sigma^{i + k - m}(a_m) = a_{i + k - m}$.
    Since $(k + i) - (i + k - m) = m$.
    Since $m \mid m$, $m \mid (k + i) - (i + k - m)$.
    This means that for $\sigma^i(a_k) = a_{i + k - m}$,
    we have $(i + k - m) \equiv (k + i) \mod m$.
    Otherwise, if we had $i + k - m > m$,
    then we could at each step repeat the process and substract m from
    the value of $c$.
    This is the equivalent of mapping $a_m$ to $a_m$ all over again,
    until we have substracted $j$ times $m$ such that $i + k - jm \leqslant m$
    (this can be shown to work using the division algorithm).
    At this point, $\sigma^i(a_k) = a_{i + k - jm}$.
    We have $(k + i) - (i + k - jm) = jm$,
    and $m \mid jm$,
    so $m \mid (k + i) - (i + k - m)$.
    This means that for $\sigma^i(a_k) = a_{i + k - jm}$,
    we have $(i + k - jm) \equiv (k + i) \mod m$.
    Which completes the proof.


    \section*{Exercise 11 $***$}
    We know that for an $m$-cycle $\sigma = (a_1, a_2, ... a_m)$,
    $\sigma^i(a_k) = a_{(k+i \mod m)}$.
    Proof that $\sigma^i$ is an $m$-cycle
    if and only if $gcd(m, i) = 1$: \\
    First we prove that $\sigma^i$ is made up of $\sfrac{m}{g}$-cycles,
    where $g = gcd(m, i)$.
    Since $g \mid m, i$, $m = bg$ and $i = cg$ for some $b,c \in \Z$.
    Consider $i \mod m = cg \mod bg \in \{0, g, 2g, ... (b-1)g\}$.
    This is because $\sfrac{cg}{bg} = \sfrac{c}{b}$,
    so by the division algorithm, $c = qb + r$
    where $0 \leqslant r < m$.
    So $c \equiv g \mod b$
    Hence $cg = qbg + gr$,
    which means that $cg \equiv gr \mod bg$, 
    meaning that $m \equiv gr \mod i$.
    Since $r \in \{0, 1, 2, ... (b-1)\}$,
    then $gr \in \{0, g, 2g, ... (b-1)g\}$,
    so $(i \mod m) \in \{0, g, 2g, ... (b-1)g\}$.
    Now consider $k \mod m$.
    Since $0 \leqslant k \leqslant m$,
    then $k \mod m = k$ when $k < m$,
    and $k \mod m = 0$ when $k = m$. \\
    We know that $(a + b) \mod c = ((a \mod c )+ (b \mod c)) \mod c$. \\
    First, for $k = m$, $k \mod m = 0$.
    This means that $(k+i) \mod m = (0 + i \mod m) \mod m = i \mod m$.
    So $(k+i) \mod m \in \{0, g, 2g, ... (b-1)g\}$. \\
    On the other hand, for $k < m$, $k \mod m = k$.
    So we have $(k + i) \mod m = (k + (i \mod m)) \mod m$. \\
    Hence $(k + i) \mod m = S$,
    where $|S| = |\{0, g, 2g, ... (b-1)g\}| = b$.
    We can show $S$ has length $b$ by proving that for each of the
    $b$ modulo classes $(i \mod m)$,  
    $((k + (i \mod m)) \mod m)$ are all distinct.  \\
    First note that, $\forall p, q \in \Z, 0 \leqslant p, q \leqslant (b-1)$,
    if $p \neq q$, then $(pg + k) \mod m \neq (qg + k) \mod m$. 
    This is because $pg + k - (qg + k) = (p - q)g$.
    We know by definition that  $m = gb$, and that $-b < p - q < b$
    since $0 \leqslant p, q < b$. So $|(p-q)| < b$,
    so $m > (p - q)g$,
    so $m \nmid (p - q)g$
    (unless $p - q = 0$, but we ruled out the possibility by assuming
    $p \neq q$).
    This means that $\forall p, q, (pg + k) \mod m \neq (qg + k) \mod m$.
    So whether $k = m$ or $k < m$,
    $(k + i) \mod m$ can be equal to $b$ values,
    where $b = \sfrac{m}{g}$, completing the initial proof. \\ 
    Now, if $g = 1$, then $b = m$, so each element $a_k$ can be mapped to
    one of $m$ values when applying $\sigma^i$, making it an $m$-cycle. \\ 
    Conversely, if $\sigma^i$ is an $m$-cycle, it must be able to map all
    elements $a_k$ to one of $m$ values, so $b = m$, meaning $g = 1$. 


    \section*{Exercise 12}
    \begin{enumerate}[label=\textbf{\alph*.}]
        \item 
            From the exercise 1.1.3.11, we determined that for an
            $m$-cycle $\sigma$, $\sigma^i$ will map any element $a_k$ 
            in a cycle of length $\dfrac{m}{gcd(i, m)}$. \\
            We have $\tau = (1\;2)(3\;4)(5\;6)(7\;8)(9\;10)$
            where $\tau = \sigma^k$ for some $k \in \Z$
            and $\sigma$ is an $n$-cycle, $n \geqslant 10$.
            Since $\sigma^i$ maps any element in a cycle of length
            $\dfrac{m}{gcd(i, m)}$,
            it stands to reason that we would have $gcd(i, m)$ cycles
            (since all are disjoint).
            So as $\tau$ has 5 2-cycles, $gcd(k, n) = 5$
            and $\dfrac{m}{gcd(k, n)} = 2$. So for $n = 10$ and $k = 5$,
            the 10-cycle $\sigma$ to the power 5, gives us $\tau$.
        \item
            This time around, we have $\tau = (1\;2\;3)(4\;5)$.
            Intuitively, it shouldn't be possible for $\tau$ to be
            equal to $\sigma^k$ for some integer $k$
            where $\sigma$ is an $n$-cycle and $n \geqslant 5$.
            This is because $\sigma^k$
            will generate only cycles of length $\dfrac{m}{gcd(k, n)}$,
            meaning they are all equal in length,
            and $\tau$ has cycles of length both 2 and 3.
            To prove this formally, 
            we begin by assuming be contradiction, that $\tau = \sigma^k$
            for $n \geqslant 5$.
            If $n > 5$, then $\sigma$ permutes the elements 6, 7...
            but we know that only way 6, 7... will be fixed is
            if $\sigma^k = 1$ (if all of the elements are fixed),
            which is contradicts our assumption that $\sigma^k = \tau$.
            So we may assume $n = 5$.
            Since there are no 5-cycles in $\tau$,
            $gcd(k, n) \neq 1$. So $gcd(k, 5) \neq 1$.
            Since 5 is prime, then k must be a multiple of 5.
            So $k = 5p$ for some $p \in \Z$.
            But then $\sigma^{5p} = (\sigma^5)^p = 1^p = 1 \neq \tau$,
            which is a contradiction. \\
            So $\tau \neq \sigma^k$ for any $\sigma$ or $k$.
    \end{enumerate}


    \section*{Exercise 13}
    Proof that for $\sigma \in S_n$, $|\sigma| = 2$
    if and only if $\sigma$ is the product of commuting (disjoint) 2-cycles: \\
    If $\sigma = (a_1\;b_1)(a_2\;b_2)...(a_n\;b_n)$,
    then $\forall a_k$,
    $\sigma^2(a_k) = \sigma(\sigma(a_k)) = \sigma(a_{k \pm 1}) = a_k$.
    So since $\sigma^2 = 1$ and $\sigma \neq 1$,
    $|\sigma| = 2$. \\
    Inversely, if $\sigma$ contains an $n$-cycle where $n \geqslant 3$,
    for example, $(a_k, a_{k+1}, a_{k+2})$,
    then $\sigma^2(a_k) = \sigma(\sigma(a_k)) = \sigma(a_{k+1}) = a_{k+2}$.
    So $|\sigma| > 2$.
    Which proves the converse by contraposition. 


    \section*{Exercise 14}
    Proof that for $\sigma \in S_n$, $|\sigma| = p$ where $p$ is a prime
    if and only if $\sigma$ is the product of commuting (disjoint)
    $p$-cycles: \\
    Repeating $n$-cycles $n$ times is the equivalent of doing nothing
    so long as the $n$-cycles are disjoint.
    If all cycles in $\sigma$ are disjoint $p$-cycles,
    then $\sigma^p = 1$.
    We know that $p$ is the smallest number to the power of which each
    $p$-cycle in $\sigma$ becomes the identity. 
    Since all the cycles are disjoint $p$-cycles,
    this property extends to $\sigma$,
    so $|\sigma| = p$. \\
    Conversely, if $|\sigma| = p$, where $p$ is the prime,
    then $\sigma^p = 1$.
    So each cycle in $\sigma$ is either a $p$-cycle or a 1-cycle.
    Since we assumed $|\sigma| = p$, and 1 isn't prime,
    we conclude that every cycle must be a $p$-cycle. \\
    To give a counterexample where $p$ isn't prime,
    we have $(1\;2)(3\;4\;5)$, which has order 6,
    but can't be written as a product of disjoint $6$-cycles (5 at most).


    \section*{Exercise 15 $***$}
    Proof that the order of an element in $S_n$ is the least common multiple
    of the size of the cycles in the cycle decomposition: \\
    The smallest number to the power of which an $m$-cycle becomes the 
    identity is $m$.
    So for a product of disjoint cycles (such as the cycle decomposition),
    for $\sigma^k = 1$, $k$ must be a multiple of the lengths of all the
    cycles. 
    By definition, the order is the smallest such number.
    So the least common multiple of all of the cycle lengths is the order of
    the element. 
    
    
    \section*{Exercise 16}
    If we have an $m$-cycle in $S_n$,
    then we're choosing $m$ elements out of $n$ and permuting them.
    That means we have $P_n^m$ permutations.
    However, for any $m$-cycle, we can shift the elements by 1, 2, 3 ... $m$
    without changing the nature of the permutation ($m$ times).
    So we have $\dfrac{P_n^m}{m}$ distinct $m$-cycles.
    \[ \dfrac{P_n^m}{m} = \dfrac{\dfrac{n!}{(n - m)!}}{m}
    = \dfrac{n!}{m(n - m)!} = \dfrac{(n)(n-1)...(n - m + 1)}{m} \]


    \section*{Exercise 17}
    For $S_n$ with $n \geqslant 4$, the number of elements that are a
    product of disjoint 2-cycles can be calculated by noting that
    for the first 2-cycle, we have $n$ and $(n-1)$ choices for each number,
    and for the second cycle, we have $(n-2)$ and $(n-3)$ choices as
    the cycles are disjoint. We can also swap between the elements in the
    cycles, and swap the two cycles together (disjoint means commutative).
    So we divide the product by $2 \times 2 \times 2$.
    So we have $\dfrac{n(n-1)(n-2)(n-4)}{8}$ cycles of this form.


    \section*{Exercise 18}
    $S_5$ contains elements of the form: \\ 
    $1$, which has order 1. \\
    $(a\;b)$, which has order 2, \\
    $(a\;b\;c)$, which has order 3, \\
    $(a\;b\;c\;d)$, which has order 4, \\
    $(a\;b\;c\;d\;e)$, which has order 5, \\
    $(a\;b\;c)(d\;e)$, which has order 6, \\
    $(a\;b)(c\;d)$, which has order 2. \\
    So $n \in \{ 1, 2, 3, 4, 5, 6 \}$.


    \section*{Exercise 19}
    $S_7$ contains elements of the form: \\ 
    1, which has order 1. \\
    $(a\;b)$, which has order 2, \\
    $(a\;b\;c)$, which has order 3, \\
    $(a\;b\;c\;d)$, which has order 4, \\
    $(a\;b\;c\;d\;e)$, which has order 5, \\
    $(a\;b\;c)(d\;e)$, which has order 6, \\
    $(a\;b)(c\;d)$, which has order 2. \\
    $(a\;b\;c)(d\;e\;f)$, which has order 3, \\
    $(a\;b\;c\;d)(e\;f\;g)$, which has order 12, \\
    $(a\;b\;c\;d)(e\;f)$, which has order 4, \\
    $(a\;b\;c\;d\;e)(f\;g)$, which has order 10, \\
    $(a\;b\;c\;d\;e\;f\;g)$, which has ordrer 7, \\
    $(a\;b)(c\;d)(e\;f)$, which has order 2. \\
    $(a\;b)(c\;d)(e\;f\;g)$, which has order 6. \\
    So $n \in \{ 1, 2, 3, 4, 5, 6, 7, 10, 12 \}$.
    

    \section*{Exercise 20 $***$}
    In $S_3$, we can generate all elements using $a = (1\;2)$
    and $b = (1\;2\;3)$: \\
    1 is the identity, \\
    $(1\;2) = a$, \\
    $(1\;2\;3) = b$, \\
    $(1\;3) = (1\;2\;3) \circ (1\;2) = ba$
    (with $(1\;2)$ performed first as it's composition), \\
    $(2\;3) = (1\;2) \circ (1\;2\;3) = ab$,
    (with $(1\;2\;3)$ performed first as it's composition), \\
    $(1\;3\;2) = (1\;2\;3) \circ (1\;2\;3) = b^2$. \\
    As for the relations, $a^2 = 1$ give us that a is a 2-cycle,
    and $b^3 = 1$ gives us that $b$ is a 3-cycle.
    We still don't know what cycle $ab$ is however, so we add the
    constraint that $(ab)^2 = 1$, or that $ab = ab^2$. \\
    So $S_3 = \langle a, b \mid a^2 = b^3 =1, ab = ab^2 \rangle$.

    
\end{document}