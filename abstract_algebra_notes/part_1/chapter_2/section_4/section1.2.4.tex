

\documentclass[12pt]{article}

\usepackage[margin=1in]{geometry}

% For using float option H that places figures exatcly where we want them
\usepackage{float}
% makes figure font bold
\usepackage{caption}
\captionsetup[figure]{labelfont=bf}
% For text generation
\usepackage{lipsum}
% For drawing
\usepackage{tikz}
% For smaller or equal sign and not divide sign
\usepackage{amssymb}
% For the diagonal fraction
\usepackage{xfrac}
% For enumerating exercise parts with letters instead of numbers
\usepackage{enumitem}
% For dfrac, which forces the fraction to be in display mode (large) e
% even in math mode (small)
\usepackage{amsmath}
% For degree sign
\usepackage{gensymb}
% For "\mathbb" macro
\usepackage{amsfonts}
\newcommand{\N}{\mathbb{N}}
\newcommand{\Z}{\mathbb{Z}}
\newcommand{\Q}{\mathbb{Q}}
\newcommand{\R}{\mathbb{R}}
\newcommand{\C}{\mathbb{C}}

% overline short italic
\newcommand{\olsi}[1]{\,\overline{{#1}}}
\newcommand{\ang}[1]{\langle #1 \rangle}

\usepackage{indentfirst}
\usetikzlibrary{shapes,positioning,fit,calc}

\usepackage{changepage} % for adjustwidth environment

\title{%
    \Huge Abstract Algebra \\
    \large by \\
    \Large Dummit and Foote \\~\\
    \huge Part 1: Group Theory \\
    \LARGE Chapter 2: Subgroups \\
    \Large Section 4: Subgroups Generated by a Subset of a Group
}
\date{2024-04-01}
\author{Michael Saba}

\begin{document}
    \pagenumbering{gobble}
    \maketitle
    \newpage
    \pagenumbering{arabic}

    In part 1.2.3, we saw that we could generate a subgroup
    of a group using a single element from the group to
    generate the subgroup,
    which will be cyclic (will only contain powers of this element). \\
    This is a special case of a more general technique,
    where we form a subgroup of a group generated by
    a subset of elements in the group
    (the resulting group may not be cyclic). 
    In the case of a cyclic subgroup,
    we just took a singleton $\{x\}$.
    The result will be the smallest subgroup containing $x$,
    (this is because a group has to be closed under multiplication,
    which means the group must contain all powers of $x$ at least,
    so the smallest subgroup containing $x$ must be at least
    as large as $\ang{x}$). \\
    This section will show how we can extend that to any subset
    of elements (even an empty set). \\

    Let $G$ be some arbitrary group and $A$ be a subset of $G$
    (which can be $\emptyset$).
    The subgroup generated by $A$ is the one that is generated
    by all the elements of $A$ obviously;
    it means it will contain all the products that can be formed by
    combinations of elements in $A$ (and their inverses).
    We need to formally define this subgroup;
    it is the smallest subgroup of $G$ containing all of $A$.
    This was trivial to show for cyclic subgroups
    (generated by a singleton).
    However, we now need to show that
    the smallest subgroup of $G$ containing $A$
    is generated by $A$
    (contains all combinations of elements in $A$
    and their inverses)
    and vice-versa
    (that the group obtained by closing $A$
    under the group operation and inverses
    is the smallest subgroup of $G$ containing $A$). \\

    In order to show this equivalence,
    we will need the following proposition:
    If $\mathcal{A}$ is any non-empty collection of subgroups of
    some group $G$,
    then the intersection of all the members of $A$
    is also a subgroup of $G$. \\
    To prove this, let's 
    \[ K = \bigcap_{H \in \mathcal{A}} H \]
    Since all $H \in \mathcal{A}$ are subgroups,
    the all contain the identity $1$,
    which means that $K$ will contain $1$,
    which means that $K \neq \emptyset$. \\
    Now for any element $a, b \in K$,
    we know that $a, b \in H$ for all $H \in \mathcal{A}$.
    Since all elements $H$ are subgroups,
    this means that must all contain $ab^{-1}$
    by closure under inverses and multiplication.
    So $ab^{-1} \in K$,
    which means that $K$ satisfies the subgroup criterion,
    so $K \leqslant G$. \\

    Now, for any subset $A$ of a group $G$,
    we can define $\ang{A}$
    to be the intersection of all subgroups of $G$
    containing $A$:
    \[ \ang{A} = 
    \bigcap_{\substack{A \subseteq H \\ H \leqslant G}} H \]
    We know that $\ang{A}$ is a subgroup of $G$ by the last proposition,
    taking
    \[ \mathcal{A} = 
    \{ H \mid H \leqslant G \text{ and } A \subseteq H \} \]
    Since $\ang{A} \leqslant G$,
    and $A \subseteq \ang{A}$
    (as $A \in H$ for all $H \in \mathcal{A}$),
    then $\ang{A} \in \mathcal{A}$.
    Since all elements of $\mathcal{A}$ also contain $\ang{A}$
    as it is their intersection,
    it must be that $\ang{A}$
    is the unique minimal element of $\ang{A}$,
    which means that $|\ang{A}| < |H|$
    for any $H \in \mathcal{A}$. \\
    We can prove it is minimal,
    by showing that $|\ang{A}| \leqslant |H|$
    for any $H \in \mathcal{A}$
    (that no subgroups in $\mathcal{A}$ are smaller than $A$).
    This is true because $\ang{A}$
    is the intersection of all subgroups in $\mathcal{A}$,
    which means that all subgroups $H \in \mathcal{A}$
    must contain $\ang{A}$.
    So $|A| \leqslant |H|$ for any element $H \in \mathcal{A}$. \\ 
    To show uniqueness,
    we need to show that $A$ is the only element
    with its order
    (no other subgroup is also minimal).
    We can prove that by showing that for any subgroup
    $H \in \mathcal{A}$,
    if $|H| = |\ang{A}|$,
    then $H = \ang{A}$.
    We know that $H$ must contain $A$
    as it is the result of intersection with $H$,
    then all elements in $A$,
    so $\ang{A} \subseteq H$.
    So if $|H| = |\ang{A}|$,
    it must be that $H = \ang{A}$,
    proving that $\ang{A}$ is the only subgroup in $\mathcal{A}$
    with its minimal order. \\
    In conclusion,
    $\ang{A}$ is the unique smallest subgroup of $G$
    containing $A$. \\

    The way we defined $\ang{A}$
    it called a \textit{top down} approach;
    it doesn't exatcly explain how we can construct $\ang{A}$,
    or find the elements in it. \\
    
    Now we will define
    the subgroup of a group $G$ generated by a subset $A$ of $G$.
    Consider the set $\olsi{A}$.
    which is the set of closure of $A$
    under the group operation and inverses.
    This means that $\olsi{A}$ will contain
    all the ways we can combine the elements of $A$
    together using inverses and the group operation,
    such that it becomes closed under them:
    \[ \olsi{A} =
    \{ a_1^{\epsilon_1}a_2^{\epsilon_2} \dots a_n^{\epsilon_n} \mid
    n \in \Z, n \geqslant 0, a_i \in A,
    \text{ and } \epsilon_i = \pm 1 \text { for each } i  \} \]
    where $\olsi{A} = \{1\}$ if $A = \emptyset$.
    Note that $n$ can be arbitrarily large;
    which means that we can have as many elements 
    and their inverses stacked in our product
    (each finite combination is called a \textbf{word}).
    So $\olsi{A}$ could very well be infinite.
    Also note that $a_i$ and $a_j$ where $i \neq j$
    need not be distinct,
    so if we want to express $a^2$, we can just write $aa$. \\
    This set is a subgroup of $G$,
    and we will prove it now. \\
    We know that even if $A$ is empty
    (which we did not prohibit)
    $\olsi{A} \neq \emptyset$.
    Moreover, if $a, b \in \olsi{A}$,
    then $a = a_1^{\epsilon_1}a_2^{\epsilon_2} \dots a_n^{\epsilon_n}$
    and $b = b_1^{\delta_1}b_2^{\delta_2} \dots b_m^{\delta_m}$,
    which means that $b^{-1} = b_m^{-\delta_m}b_{m-1}^{-\delta_{m-1}}
    \dots b_1^{-\delta_1}$.
    We can thus write
    \[ ab^{-1} = 
    a_1^{\epsilon_1}a_2^{\epsilon_2} \dots a_n^{\epsilon_n}
    b_m^{-\delta_m}b_{m-1}^{-\delta_{m-1}} \dots b_1^{-\delta_1} \]
    where all elements $a_i$ and $b_i$ are in $A$,
    and all elements $\epsilon_i$ and $\delta_i$ are $\pm 1$.
    Since $ab^{-1}$ is a combination of elements in $A$
    and their inverses,
    it must be part of $\olsi{A}$ by its definition.
    So we can conclude $\olsi{A}$ is a subgroup of $G$. \\
    It is clear to see that $\olsi{A}$
    is the subgroup of $G$ generated by the subset $A$
    (since the elements of $A$ literally generate it). \\

    Now, we've come to the point where we can show that
    the subgroup of some group $G$
    generated by a subset $A$ of $G$ 
    is the smallest subgroup containing $A$.
    This means that we need to show that $\ang{A} = \olsi{A}$. \\
    Intuitively, this makes perfect sense;
    we know that a subgroup has to be closed under inverses
    and the group operation,
    so if a subgroup contains $A$,
    then it must contain all products and inverse combinations
    of elements in $A$ at least.
    If such a subgroup is the minimal subgroup containing $A$
    (which is $\ang{A}$),
    then it must have enough elements to be closed under $A$
    and no more,
    which means that it must contain all combinations of elements
    in $A$,
    but have no other elements thrown in
    (which is how we defined $\olsi{A}$). \\
    Now, we need to formally prove that $\ang{A} = \olsi{A}$.
    We know that all elements $a \in A$
    are contained in $\olsi{A}$
    as they can all be written as $a^1$
    (a combination of elements of $A$),
    which means that $A \subseteq \olsi{A}$.
    Since $\olsi{A} \leqslant G$,
    and $A \subseteq \olsi{A}$,
    we now know that $\olsi{A} \in \mathcal{A}$. 
    By definition,
    we know that $\ang{A}$ is the intersection of all 
    subgroups in $\mathcal{A}$,
    including $\olsi{A}$.
    so $\ang{A} \subseteq \olsi{A}$. \\
    On the other hand,
    we know that $\ang{A}$ contains all elements of $A$
    since it is the intersection of subgroups that all contain $A$.
    But we also know that $\ang{A} \leqslant G$,
    which means that it must be closed under its
    group operation and inverses.
    So $\ang{A}$ must contain all product
    combinations of elements in $A$ and their inverses,
    which means that any word
    $a_1^{\epsilon_1}a_2^{\epsilon_2} \dots a_n^{\epsilon_n}$
    where $a_i \in A$
    must be in $\ang{A}$.
    This means that all elements of $\olsi{A}$
    are also in $\ang{A}$,
    which means that $\olsi{A} \subseteq \ang{A}$. \\
    Thus we can conclude that $\olsi{A} = \ang{A}$. \\

    At this point,
    we can use $\olsi{A}$ and $\ang{A}$ interchangebly
    to refer to the subgroup of $G$
    that is generated by its subset $A$
    (which contains all the elements
    generated by combining elements in $A$ and their inverses,
    no more).
    We will use $\ang{A}$ from now on to refer to it. \\
    
    We can modify $\olsi{A}$'s
    definition a little bit to group consecutive
    $a_i$s together if they are the same element
    by just adding their powers.
    So if we have $a^1a^1a^1a^{-1}a^1$,
    we can just write it as $a^3$.
    So we get
    \[ \ang{A} = \olsi{A} 
    = \{ a_1^{\epsilon_1}a_2^{\epsilon_2} \dots a_n^{\epsilon_n} \mid
    n \in \Z, n \geqslant 0, a_i \in A,
    a_i \neq a_{i+1},
    \text{ and } \epsilon_i \in \Z \text { for each } i \} \]
    Note that that means that there won't
    be any consecutive elements $a_i$ and $a_j$
    that are the same anymore;
    if they were the same,
    then we would have just combined them.
    This does not however mean that all elements $a_i$
    in each combination are distinct.
    This approach of grouping similar elements as one power
    only applies to consecutive elements that are the same;
    since $G$ may not be abelian,
    we can't just move non-consecutive
    elements that are the same to group them in one power.
    So the same element may appear more than once,
    just not consecutively. \\

    On the other hand, if $G$ were abelian,
    then we could just rearrange each word in $\olsi{A}$
    so that the same elements are all consecutive
    (since they all commute).
    We could then group them in one power of that element,
    and have no repetitions whatsoever,
    not just consecutively. \\
    This means that we can write each combination in $\olsi{A}$
    as the integer powers of the elements of $A$,
    where each element is always present exatcly once
    (the power may be 0
    if the element is not a factor in the combination). \\
    So if $A = \{ a_1, a_2 \dots a_k \}$,
    then we would have
    \[ \ang{A} = \olsi{A} 
    = \{ a_1^{\alpha_1}a_2^{\alpha_2} \dots a_k^{\alpha_k} \mid
    \alpha_i \in \Z \text{ for each } i \} \]
    Now, if we assueme that all of the elements of $A$
    also have finite order $d_i = |a_i|$,
    then in all of our combinations in $\olsi{A}$,
    we will never have to raise the power to more than
    $d_i$,
    as the values will just repeat
    (that applies to all negative and positive powers).
    This means that at most,
    we will have $d_1d_2\dots d_k$ combinations
    that we can make without repeating powers of elements.
    So $|\olsi{A}| \leqslant d_1d_2\dots d_k$
    is an upper bound on the size of $\olsi{A}$.
    Note that this is only a bound, and not the exact size,
    as different configurations may still lead to the same element;
    guaranteeing that we never have the same 
    combination of powers of $a_i$
    (by restricting the exponents to be at most $d_i$)
    does not guarantee that the final result of the combinations
    will be distinct
    (it's possible that $a^\alpha b^\beta = a^\gamma b^\delta$
    even if $a^\alpha \neq a^\gamma$ and  $b^\beta \neq b^\delta$). \\
    When $G$ is not abelian,
    it is much more complex,
    and we can't bound the order of $\olsi{A}$
    even if all of the elements in $A$ are finite and have
    well known orders.
    This is because the combinations of elements in $\olsi{A}$
    cannot be furtehr simplified and grouped,
    as the elements don't commute;
    this means that unlike the case where $G$ was abelian,
    the same elements $a_i$
    may repeat a different amount of times in each configuration
    instead of always appearing exactly once.
    So if $|A| \geqslant 2$,
    it can be very difficult to find the order of
    the subgroup $A$ generates.
    When $|A| = 1$ or $|A| = 0$ however,
    this is just the case where the group is cyclic or trivial,
    where we can always easily calculate the order. \\

    If $A = \{ a_1, a_2 \dots a_n\}$,
    we can write $\ang{A}$ as $\ang{a_1, a_2 \dots a_n}$
    instead of $\ang{ \{a_1, a_2 \dots a_n\}}$. \\
    When we have a subgroup of a group $G$
    that is generated by two subsets of $G$ $A$ and $B$,
    it just means that the generators can come from $A$
    or $B$,
    so it really just means that $A \cup B$
    is the subset of $G$ that generates the subgroup.
    Instead of writing $\ang{A \cup B}$ though,
    we can write $\ang{A, B}$;
    it's still technically just one subset
    $A \cup B$ that generates the subgroup,
    so the same rules we learned all still apply. \\ 


\end{document}
