
\documentclass[12pt]{article}

\usepackage[margin=1in]{geometry}

% For vertical brace rcases
\usepackage{mathtools}
% For positioning figures
\usepackage{float}
% makes figure font bold
\usepackage{caption}
\captionsetup[figure]{labelfont=bf}
% For text generation
\usepackage{lipsum}
% For drawing
\usepackage{tikz}
% For manipulating coordinates
\usetikzlibrary{calc}
% For smaller or equal sign and not divide sign
\usepackage{amssymb}
% For the diagonal fraction
\usepackage{xfrac}
% For enumerating exercise parts with letters instead of numbers
\usepackage{enumitem}
% For dfrac, which forces the fraction to be in display mode (large) e
% even in math mode (small)
\usepackage{amsmath}
% For degree sign
\usepackage{gensymb}
% For "\mathbb" macro
\usepackage{amsfonts}
\newcommand{\N}{\mathbb{N}}
\newcommand{\Z}{\mathbb{Z}}
\newcommand{\Q}{\mathbb{Q}}
\newcommand{\R}{\mathbb{R}}
\newcommand{\C}{\mathbb{C}}
\newcommand{\F}{\mathbb{F}}

% overline short italic
\newcommand{\olsi}[1]{\,\overline{\!{#1}}}

\title{%
    \Huge Abstract Algebra \\
    \large by \\
    \Large Dummit and Foote \\~\\
    \huge Part 1: Group Theory \\
    \LARGE Chapter 1: Introduction to Groups \\
    \Large Section 4: Matrix Groups
}
\date{2023-07-14}
\author{Michael Saba}

\begin{document}
    \pagenumbering{gobble}
    \maketitle
    \newpage
    \pagenumbering{arabic}


    \section*{Exercise 1}
    Consider the field $\Z/2\Z$ denoted by $\F_2$. \\
    Since $\F_2 = \{\olsi{0}, \olsi{1}\}$,
    then we have $2^4 = 16$ different ways of creating a 2 by 2 matrix out of its
    elements.
    However, for the elements to belong to the group $GL_2(\F_2)$,
    they must have a determinant that's not 0.
    So $ad - bc \neq 0$. This means that: \\
    $a, b, c, d$ can't all be 1,
    and if either $a$ or $d$ were 0, $b$ and $c$ can't be 0.
    Likewise, if either $b$ or $c$ were 0, $a$ and $b$ can't be 0.
    This eliminates $1 + 9 = 10$ of the configurations, leaving 6.
    So $|GL_2(\F_2)| = 6$.


    \section*{Exercise 2}
    $GL_2(\F_2)$ contains the following elements: \\
    \[ \begin{bmatrix}
        1 & 0 \\
        0 & 1 \\
    \end{bmatrix},
    \begin{bmatrix}
        1 & 1 \\
        1 & 0 \\
    \end{bmatrix},
    \begin{bmatrix}
        1 & 1 \\
        0 & 1 \\
    \end{bmatrix},
    \begin{bmatrix}
        1 & 0 \\
        1 & 1 \\
    \end{bmatrix},
    \begin{bmatrix}
        0 & 1 \\
        1 & 1 \\
    \end{bmatrix},
    \begin{bmatrix}
        0 & 1 \\
        1 & 0 \\
    \end{bmatrix} \]
    which respectively, have orders 2, 3, 3, 3, 3, 2.


    \section*{Exercise 3}
    We have:
    \[ \begin{bmatrix}
        1 & 1 \\
        1 & 0 \\
    \end{bmatrix} \cdot
    \begin{bmatrix}
        1 & 1 \\
        0 & 1 \\
    \end{bmatrix}
    = \begin{bmatrix}
        1 & 0 \\
        0 & 1 \\
    \end{bmatrix} \]
    and: 
    \[ \begin{bmatrix}
        1 & 1 \\
        0 & 1 \\
    \end{bmatrix} \cdot
    \begin{bmatrix}
        1 & 1 \\
        1 & 0 \\
    \end{bmatrix}
    = \begin{bmatrix}
        1 & 0 \\
        1 & 1 \\
    \end{bmatrix} \]
    So $GL_2(\F_2)$ is not abelian.


    \section*{Exercise 4}
    Proof that if $n$ isn't prime, $\Z/n\Z$ isn't a field: \\
    Assume that $\Z/n\Z$ is a field and $n$ is composite.
    Then there exists an element $\olsi(a) \in (\Z/n\Z)^\times$
    such that $gcd(a, n) > 1$.
    If this is the case,
    then there is a modulo class $\olsi{b} \in (\Z/n\Z)^\times$
    such that $\olsi{a} \cdot \olsi{b} = \olsi{n} = \olsi{0}$.
    Since $(\Z/n\Z)^\times$ (the multiplicative group of $\Z/n\Z$)
    is by definition $\Z/n\Z - \{0\}$ (the additive identity),
    then our demonstartion shows that $(\Z/n\Z)^\times$ is not closed
    under multiplication, which is a contradition since it's supposed to
    be an (abelian) group, completing the proof.


    \section*{Exercise 5}
    Proof that $GL_n(F)$ is finite if an only if $F$ is finite
    (where $F$ is a field and $n \in \Z^+$): \\
    If $F$ is finite, we can assume $|F| = m < \infty$
    then for each element in the $n$ by $n$ matrix, we have $m$ options.
    So we can combine the elements in the field in $m^{n^2}$ ways.
    Not all of these matrices are invertible,
    so this is just an upper limit,
    but it does show that $GL_n(F)$ has at most $m^{n^2}$ elements,
    making it finite. \\
    Conversely, if $F$ is infinite,
    then consider the susbet of $GL_n(F)$ made up of matrices of the form:
    \[\begin{bmatrix}
    f & 0 & 0 & \dots & 0\\
    0 & 1 & 0 & \dots & 0 \\
    0 & 0 & 1 & & \vdots \\
    \vdots & \vdots & & \ddots & 0\\ 
    0 & 0 & \dots & 0 & 1 \\
    \end{bmatrix}_{n \times n}\]
    where $f \in F - \{0\}$ and 1 and 0 are the multiplicative and
    additive identities respectively. \\
    Since this matrices of this form are diagonal,
    their determinants are the products of their diagonal elements,
    which are always equal to $1 \cdot 1 ... 1 \cdot f = f$.
    Since $f \neq 0$, the matrices are always invertible, meaning
    they always belong to $GL_n(F)$.
    So since $F$ is infinite, there must an infinite number of
    such matrices.
    This a lower limit, so the order of $GL_n(F)$ must be infinite.


    \section*{Exercise 6}
    Proof that if $|F| = q < \infty$,
    then $|GL_n(F)| < q^{n^2}$ for some $n \in \Z^+$: \\
    Since there are $q$ elements in $F$,
    and $n \times n$ elements in a matrix in $GL_n(F)$, 
    then we can have $q^{n^2}$ unique matrix configurations.
    Since at least some of these matrices are singular, such as:
    \[\begin{bmatrix}
    0 & \dots & 0\\
    \vdots & \ddots & \vdots \\ 
    0 & \dots & 0 \\
    \end{bmatrix}_{n \times n}\]
    then $|GL_n(F)| < q^{n^2}$.


    \section*{Exercise 7 $***$}
    For $\F_p = \Z/p\Z$ where $p$ is prime,
    we have $|GL_2(\F_2)| = p^4 - p^3 - p^2 + p$. \\
    To show this, we first note that $|F_p|$ = p. 
    So the number of unique matrices that can be generated by $F_p$
    is $p^4$, providing us an upper limit.
    We know that
    \[ GL_2(\F_2) = \left\{
        \begin{bmatrix}
            a & b \\
            c & d \\
        \end{bmatrix} 
        \; \middle\vert \;
        a, b, c, d \in \F_2,
        ad - bc \neq 0  
    \right\} \] 
    where 0 is the additive identity of $\F_2$.
    Now we consider all the configurations that give a singular matrix: \\
    When $a = 0$,
    then the matrix is singular if $bc = 0$,
    which means that $b = 0$ or $c = 0$.
    This means there are $p$ options for both $b$ and $c$,
    which we must add since either of them at any time is fixed as 0.
    We also subtract the option of $b = 0$ and $c = 0$ from the sum
    since it's repeated twice.
    In this scenario, $d$ can be any element,
    so $d$ has $p$ options,
    and we multiply the result by it,
    since the values of $b$ and $c$ are independent of $d$.
    So we have $(2p - 1)p$ options. \\
    When $a \neq 0$,
    to get a singular matrix, we need $ad - bc = 0$,
    so we have $d = a^{-1}bc$.
    So $a$, $b$, and $c$ can have any values,
    but $a \neq 0$,
    and with that $d$ is always fixed.
    So we have $p$ options for $b$ and $c$,
    and $(p-1)$ options for $a$,
    and no further conditions or dependecies,
    so we use the multiplication rule to get $p^2(p-1)$ options. \\
    The above described matrices, are singular,
    so we must subtract their configurations from our upper limit. \\
    We get $|GL_2(\F_2)| = p^4 - ((2p - 1)p + p^2(p-1))
    = p^4 - (2p^2 - p + p^3 - p^2)
    = p^4 - p^3 - p^2 + p$


    \section*{Exercise 8 $***$}
    Proof that $GL_n(F)$ is non-abelian for any field $F$
    and $n \in \Z^+$ such that $n \geqslant 2$: \\
    For $n = 2$, we've already shown $GL_2(F)$ is not abelian in exercise
    1.1.4.3, since  
    \[ \begin{bmatrix}
        1 & 1 \\
        1 & 0 \\
    \end{bmatrix} \cdot
    \begin{bmatrix}
        1 & 1 \\
        0 & 1 \\
    \end{bmatrix}
    = \begin{bmatrix}
        1 & 0 \\
        0 & 1 \\
    \end{bmatrix} \]
    and: 
    \[ \begin{bmatrix}
        1 & 1 \\
        0 & 1 \\
    \end{bmatrix} \cdot
    \begin{bmatrix}
        1 & 1 \\
        1 & 0 \\
    \end{bmatrix}
    = \begin{bmatrix}
        1 & 0 \\
        1 & 1 \\
    \end{bmatrix} \]
    where 1 and 0 are the multiplicative and additive identities of
    $F$ respectively. \\
    Now take 
    \[ A = \begin{bmatrix}
        1 & 1 \\
        1 & 0 \\
    \end{bmatrix}\]
    and
    \[ B = \begin{bmatrix}
        1 & 1 \\
        0 & 1 \\
    \end{bmatrix} \]
    For any $n \geqslant 3$,
    and any 2 matrices $C$ and $D$,
    we know this about block matrices that contain them
    \[ \begin{bmatrix}
        C & 0 \\
        0 & 1 \\
    \end{bmatrix} \cdot
    \begin{bmatrix}
        D & 0 \\
        0 & 1 \\
    \end{bmatrix}
    = \begin{bmatrix}
        C \cdot D & 0 \\
        0 & 1 \\
    \end{bmatrix} \]
    where $C \cdot D$ is resolved via matrix multiplication. \\
    We also know that if $C$ and $D$ aren't singular,
    then neither are
    \[ \begin{bmatrix}
        C & 0 \\
        0 & 1 \\
    \end{bmatrix}, 
    \begin{bmatrix}
        D & 0 \\
        0 & 1 \\
    \end{bmatrix} \]
    since they are diagonal block matrices, where their diagonals neither
    contain 0 nor matrices with determinant equal to 0. \\
    We can now use induction to prove that there is at least one pair of
    elements in each matrix group $GL_n(F)$ that don't commute: \\
    \textbf{Basis step:}
    We know that for $n = 3$, we can define
    \[ A_3 =
    \begin{bmatrix}
    A & 0 \\
    0 & 1 \\
    \end{bmatrix}\]
    and 
    \[ B_3 =
    \begin{bmatrix}
    B & 0 \\
    0 & 1 \\
    \end{bmatrix} \]
    We know that $AB \neq BA$,
    so $A_3B_2 \neq B_3A_3$ because
    \[ A_3B_3 = 
    \begin{bmatrix}
    AB & 0 \\
    0 & 1 \\
    \end{bmatrix}\]
    and 
    \[ B_3A_3 =
    \begin{bmatrix}
    BA & 0 \\
    0 & 1 \\
    \end{bmatrix} \]
    \textbf{Inductive hypothesis:} Assume that for $n = k$,
    $A_kB_k \neq B_kA_k$. \\
    \textbf{Inductive step:} For $n = k + 1$, we have
    \[ A_{k+1} =
    \begin{bmatrix}
    A_k & 0 \\
    0 & 1 \\
    \end{bmatrix}\]
    and 
    \[ B_{k+1} =
    \begin{bmatrix}
    B_k & 0 \\
    0 & 1 \\
    \end{bmatrix} \]
    Since we assumed $A_kB_k \neq B_kA_k$,
    then $A_{k+1}B_{k+1} \neq B_{k+1}A_{k+1}$.
    This is the case because:
    \[ A_{k+1}B_{k+1} = 
    \begin{bmatrix}
    A_kB_k & 0 \\
    0 & 1 \\
    \end{bmatrix}\]
    and 
    \[ B_{k+1}A_{k+1} =
    \begin{bmatrix}
    B_kA_k & 0 \\
    0 & 1 \\
    \end{bmatrix}\]
    This proves that every matrix group $GL_n(F)$ has at least one
    pair of elements (in our case, $A_n$ and $B_n$) that don't commute.
    So no group $GL_n(F)$ can be abelian.


    \section*{Exercise 9}
    Proof that multiplication is associative in $GL_n(\R)$,
    where $\R$ is the field of real numbers and $n \in \Z^+$: \\
    For $a, b, c, d, e, f, g, h, i, j, k, l \in \R$, take
    \[ A =
    \begin{bmatrix}
    a & b \\
    c & d \\
    \end{bmatrix}, \;
    B =
    \begin{bmatrix}
    e & f \\
    g & h \\
    \end{bmatrix}, \;
    C =
    \begin{bmatrix}
    i & j \\
    k & l \\
    \end{bmatrix} \]
    We have
    \[ (AB)C =
    \left( \begin{bmatrix}
    a & b \\
    c & d \\
    \end{bmatrix}
    \begin{bmatrix}
    e & f \\
    g & h \\
    \end{bmatrix} \right)
    \begin{bmatrix}
    i & j \\
    k & l \\
    \end{bmatrix} 
    = \begin{bmatrix}
    ae + fc & be + fd \\
    ag + hc & bg + dh \\
    \end{bmatrix}
    \begin{bmatrix}
    e & f \\
    g & h \\
    \end{bmatrix} \]
    \[ = \begin{bmatrix}
    iae + ifc + jag + jhc & ibe + ifd + jbg + jdh \\
    kae + kfc + lag + lhc & kbe + kfd + lbg + ldh \\
    \end{bmatrix} \]
    and 
    \[ A(BC) =
    \begin{bmatrix}
    a & b \\
    c & d \\
    \end{bmatrix}
    \left( \begin{bmatrix}
    e & f \\
    g & h \\
    \end{bmatrix}
    \begin{bmatrix}
    i & j \\
    k & l \\
    \end{bmatrix} \right)
    = \begin{bmatrix}
    a & b \\
    c & d \\
    \end{bmatrix}
    \begin{bmatrix}
    ei + jg & if + jh \\
    ke + lg & kf + kh \\
    \end{bmatrix} \]
    \[ = \begin{bmatrix}
    iae + ifc + jag + jhc & ibe + ifd + jbg + jdh \\
    kae + kfc + lag + lhc & kbe + kfd + lbg + ldh \\
    \end{bmatrix} \]
    So $\forall A, B, C \in GL_2(\R)$, $(AB)C = A(BC)$,
    which makes $GL_2(\R)$ associative.


    \section*{Exercise 10}
    For $G = \left\{ \begin{bmatrix} a & b \\ 0 & c \end{bmatrix} \;
    \middle\vert \; a, b, c \in \R, \text{ and } a, c \neq 0 \right\}$:
    \begin{enumerate}[label=\textbf{\alph*.}]
        \item 
            For all $a_1, b_1, c_1, a_2, b_2, c_2 \in \R$, we have
            \[ \begin{bmatrix}
            a_1 & b_1 \\
            0 & c_1 \\
            \end{bmatrix}
            \begin{bmatrix}
            a_2 & b_2 \\
            0 & c_2 \\
            \end{bmatrix}
            = \begin{bmatrix}
            a_1a_2 & a_1b_2 + b1_c2 \\
            0 & c_1c_2 \\
            \end{bmatrix} \]
            where $a_1a_2$, $a_1b_2 + b_1c_2$, and $c_1c_2 \in \R$.
            So the product is in $G$,
            which means $G$ is closed under multiplication.
        \item
            Since $a, c \neq 0$, $ac \neq 0$,
            so $ac - b \cdot 0 \neq 0$.
            Since the determinant of all matrices is not 0, that means all
            have inverses:
            \[ \begin{bmatrix}
            a & b \\
            0 & c \\
            \end{bmatrix}^{-1}
            = \dfrac{1}{ac}\begin{bmatrix}
            c & -b \\
            0 & a \\
            \end{bmatrix}
            = \begin{bmatrix}
            \sfrac{1}{a} & \sfrac{-b}{ca} \\
            0 & \sfrac{1}{c} \\
            \end{bmatrix} \]
            Since $\sfrac{1}{a}$, $\sfrac{-b}{ca}$,
            and $\sfrac{1}{c} \in \R$,
            that means the inverse is in $G$ for any matrix in $G$,
            so $G$ is closed under inverses.
        \item
            We know $G$ is closed under inverses and multiplication.
            We also know $G$ is not empty since the identity of $G$
            and $GL_2(\R)$ is in $G$ (the identity matrix $I_2$).
            So all we have left to show is that $G$ is a subset of $GL_2(\R)$.
            We know from part b that the matrices $G$ are invertible,
            so by definition their determinants are not 0.
            So all elements in $G$ are also in $GL_2(\R)$,
            hence $G \subseteq GL_2(\R)$,
            which means that $G \leqslant GL_2(\R)$.
        \item
            Proof that the set
            $G = \left\{ \begin{bmatrix} a & b \\ 0 & c \end{bmatrix} \;
            \middle\vert \; a, b, c \in \R, a, c \neq 0,
            \text{ and } a = c \right\}$
            is also a subrgoup of $GL_2(\R)$: \\
            We know from part a that
            \[ \begin{bmatrix}
            a_1 & b_1 \\
            0 & a_1 \\
            \end{bmatrix}
            \begin{bmatrix}
            a_2 & b_2 \\
            0 & a_2 \\
            \end{bmatrix}
            = \begin{bmatrix}
            a_1a_2 & a_1b_2 + b1_a2 \\
            0 & a_1a_2 \\
            \end{bmatrix} \]
            so as $a_1a_2 = a_1a_2$,
            this means that $G$ is closed under multiplication. \\
            Moreover, we know from part b that
            \[ \begin{bmatrix}
            a & b \\
            0 & a \\
            \end{bmatrix}^{-1}
            = \dfrac{1}{a^2}\begin{bmatrix}
            a & -b \\
            0 & a \\
            \end{bmatrix}
            = \begin{bmatrix}
            \sfrac{1}{a} & \sfrac{-b}{a^2} \\
            0 & \sfrac{1}{a} \\
            \end{bmatrix} \]
            so as $\sfrac{1}{a} = \sfrac{1}{a}$,
            this means that $G$ is closed under inverses. \\
            Since the identity matrix $I_2 \in G$ since $1 = 1$,
            $G \neq \emptyset$,
            and we know from part c that any matrices of this form are
            subsets of $GL_2(\R)$ regardless of whether or not $a = c$.
            So $G \subseteq GL_2(\R)$,
            which means that $G \leqslant GL_2(\R)$.
    \end{enumerate}  
    
    
    \section*{Exercise 11}
    For a field $F$,
    let \[H(F) = \left\{
    \begin{bmatrix} 1 & a & b \\ 0 & 1 & c \\ 0 & 0 & 1 \end{bmatrix}
    \; \middle\vert \; a, b, c \in F \right\}\]
    be the \textit{Heisenberg group over $F$} (where 0 and 1 are the
    additive and multiplicative identities of $F$ respectively). \\
    \begin{enumerate}[label=\textbf{\alph*.}]
        \item 
            For
            \[ X =
            \begin{bmatrix}
                1 & a & b \\
                0 & 1 & c \\
                0 & 0 & 1 \\
            \end{bmatrix}
            \]
            and
            \[ Y =
            \begin{bmatrix}
                1 & d & e \\
                0 & 1 & f \\
                0 & 0 & 1 \\
            \end{bmatrix} \]
            where $a, b, c, d, e, f \in F$,
            we have:
            \[ XY =
            \begin{bmatrix}
                1 & a & b \\
                0 & 1 & c \\
                0 & 0 & 1 \\
            \end{bmatrix}
            \begin{bmatrix}
                1 & d & e \\
                0 & 1 & f \\
                0 & 0 & 1 \\
            \end{bmatrix}
            = \begin{bmatrix}
                1 & a+d & b + af + e \\
                0 & 1 & c+f \\
                0 & 0 & 1 \\
            \end{bmatrix} \]
            Since $F$ is a field,
            it is closed unde multiplication and addition, 
            so $a+d$, $b + af + e$, and $c+f \in F$,
            which means $XY \in H(F)$,
            so $H(F)$ is closed under multiplication. \\
            Moreover, we have
            \[ \begin{bmatrix}
                1 & 2 & 3 \\
                0 & 1 & 4 \\
                0 & 0 & 1 \\
            \end{bmatrix}
            \begin{bmatrix}
                1 & 3 & 2 \\
                0 & 1 & 4 \\
                0 & 0 & 1 \\
            \end{bmatrix}
            = \begin{bmatrix}
                1 & 5 & 13 \\
                0 & 1 & 8 \\
                0 & 0 & 1 \\
            \end{bmatrix} \]
            and 
            \[ \begin{bmatrix}
                1 & 3 & 2 \\
                0 & 1 & 4 \\
                0 & 0 & 1 \\
            \end{bmatrix}
            \begin{bmatrix}
                1 & 2 & 3 \\
                0 & 1 & 4 \\
                0 & 0 & 1 \\
            \end{bmatrix}
            = \begin{bmatrix}
                1 & 5 & 15 \\
                0 & 1 & 4 \\
                0 & 0 & 1 \\
            \end{bmatrix} \]
            Since both matrices belong in $H(F)$ and don't commute,
            it means that $H(F)$ is not abelian.
        \item
            For all $a, b, c \in F$,
            \[ \begin{bmatrix}
                1 & a & b \\
                0 & 1 & b \\
                0 & 0 & 1 \\
            \end{bmatrix}^{-1}
            = \begin{bmatrix}
                1 & -a & ac-b \\
                0 & 1 & -c \\
                0 & 0 & 1 \\
            \end{bmatrix} \]
            is an explicit formula for the inverse of a matrix in $H(F)$
            which can obtain using Guass-Jordan elimination
            ($-a$ is the additive inverse of $a$, same for $b$ and $c$).
            Since $-a$, $ac - b$, and $-c \in F$, as the field is closed
            under addition, multiplication, and inverses, 
            this means that the inverse is in $H(F)$,
            which makes $H(F)$ closed under inverses.
        \item
            For all $a, b, c, d, e, f, g, h, i \in F$,
            \[ \left( \begin{bmatrix}
                1 & a & b \\
                0 & 1 & c \\
                0 & 0 & 1 \\
            \end{bmatrix}
            \begin{bmatrix}
                1 & d & e \\
                0 & 1 & f \\
                0 & 0 & 1 \\
            \end{bmatrix} \right) \cdot
            \begin{bmatrix}
                1 & g & h \\
                0 & 1 & i \\
                0 & 0 & 1 \\
            \end{bmatrix}
            = \begin{bmatrix}
                1 & a + d & b + e + af \\
                0 & 1 & c + f \\
                0 & 0 & 1 \\
            \end{bmatrix}
            \begin{bmatrix}
                1 & g & h \\
                0 & 1 & i \\
                0 & 0 & 1 \\
            \end{bmatrix} \]
            \[ = \begin{bmatrix}
                1 & a + d + g & b + e + h + i(a + d) + af \\
                0 & 1 & c + f + i \\
                0 & 0 & 1 \\
            \end{bmatrix} \]
            and
            \[\begin{bmatrix}
                1 & a & b \\
                0 & 1 & c \\
                0 & 0 & 1 \\
            \end{bmatrix}
            \left( \begin{bmatrix}
                1 & d & e \\
                0 & 1 & f \\
                0 & 0 & 1 \\
            \end{bmatrix} \cdot
            \begin{bmatrix}
                1 & g & h \\
                0 & 1 & i \\
                0 & 0 & 1 \\
            \end{bmatrix} \right) 
            = \begin{bmatrix}
                1 & a & b \\
                0 & 1 & c \\
                0 & 0 & 1 \\
            \end{bmatrix}
            \begin{bmatrix}
                1 & d + g & e + h + di \\
                0 & 1 & f + i \\
                0 & 0 & 1 \\
            \end{bmatrix} \]
            \[ = \begin{bmatrix}
                1 & a + d + g & b + e + h + a(f + i) + di \\
                0 & 1 & c + f + i \\
                0 & 0 & 1 \\
            \end{bmatrix} \]
            Since addition and multiplication are associative in the field
            $F$, 
            we have $ b + e + h + i(a + d) + af = b + e + h + a(f + i) + di$,
            so both products are equal for any elements in $F$,
            which makes $H(F)$ associative. \\
            This means that $H(F)$ is a group, as it is associative
            under a group operation,
            closed under that operation and inverses,
            and has an identity,
            which is the identity matrix $I_3$.
            Furthermore, since for any element in $H(F)$,
            $a_{12}$, $a_{13}$, and $a_{23}$ can be any element in $F$,
            by the multiplication rule,
            we have $|F|^3$ distinct matrices in $H(F)$,
            so $|H(F)| = |F|^3$.
        \item
            The group $H(\Z/2\Z) = H(\F_2)$ contains the following elements:
            \[ \begin{bmatrix}
                \olsi{1} & \olsi{1} & \olsi{1} \\
                \olsi{0} & \olsi{1} & \olsi{1} \\
                \olsi{0} & \olsi{0} & \olsi{1} \\
            \end{bmatrix},
            \begin{bmatrix}
                \olsi{1} & \olsi{1} & \olsi{0} \\
                \olsi{0} & \olsi{1} & \olsi{1} \\
                \olsi{0} & \olsi{0} & \olsi{1} \\
            \end{bmatrix},
            \begin{bmatrix}
                \olsi{1} & \olsi{1} & \olsi{1} \\
                \olsi{0} & \olsi{1} & \olsi{0} \\
                \olsi{0} & \olsi{0} & \olsi{1} \\
            \end{bmatrix},
            \begin{bmatrix}
                \olsi{1} & \olsi{0} & \olsi{1} \\
                \olsi{0} & \olsi{1} & \olsi{1} \\
                \olsi{0} & \olsi{0} & \olsi{1} \\
            \end{bmatrix}, \]
            \[ \begin{bmatrix}
                \olsi{1} & \olsi{0} & \olsi{1} \\
                \olsi{0} & \olsi{1} & \olsi{0} \\
                \olsi{0} & \olsi{0} & \olsi{1} \\
            \end{bmatrix},
            \begin{bmatrix}
                \olsi{1} & \olsi{1} & \olsi{0} \\
                \olsi{0} & \olsi{1} & \olsi{0} \\
                \olsi{0} & \olsi{0} & \olsi{1} \\
            \end{bmatrix},
            \begin{bmatrix}
                \olsi{1} & \olsi{0} & \olsi{0} \\
                \olsi{0} & \olsi{1} & \olsi{1} \\
                \olsi{0} & \olsi{0} & \olsi{1} \\
            \end{bmatrix},
            \begin{bmatrix}
                \olsi{1} & \olsi{0} & \olsi{0} \\
                \olsi{0} & \olsi{1} & \olsi{0} \\
                \olsi{0} & \olsi{0} & \olsi{1} \\
            \end{bmatrix} \]
            where $I_3$ (last element) is the identity, and where they
            have an order of 4, 4, 2, 2, 2, 2, 2, 1 respectively.
        \item
            Proof that every non-identity element of $H(\R)$ has infinite
            order: \\
            For all $a, b, c \in \R$, we have :
            \[ \begin{bmatrix}
                1 & a & b \\
                0 & 1 & b \\
                0 & 0 & 1 \\
            \end{bmatrix}^2
            = \begin{bmatrix}
                1 & a & b \\
                0 & 1 & b \\
                0 & 0 & 1 \\
            \end{bmatrix}
            \begin{bmatrix}
                1 & a & b \\
                0 & 1 & c \\
                0 & 0 & 1 \\
            \end{bmatrix}
            = \begin{bmatrix}
                1 & 2a & 2b + ac \\
                0 & 1 & 2c \\
                0 & 0 & 1 \\
            \end{bmatrix} \]
            If we assume the matrix we started with was not the identity $I_3$,
            then at least one of either $a$, $b$, or $c$ is not equal to 0.
            So $2a$, $2b$, or $2c$ are not 0.
            If however, $2b + ac = 0$,
            then if $2b = 0$, $ac = 0$,
            so $b = 0$ and either $a$ or $c$ are 0, but not both,
            so one of the entries $2a$ and $2c$ is not 0.
            If $2b \neq 0$, then neither $a$ nor $c$ are equal to 0, and
            both entries $2a$ and $2c$ won't be 0. \\
            We can repeat this argument and raise any matrix to any
            arbitrary power,
            but as long as the product produces a matrix with at least one
            of the three entries that isn't 0, the product will have 
            that same property. \\
            So the matrices in $H(\R)$ have infinite order as they can't
            be turned into the identity by raising them to a power different
            from 0. 
    \end{enumerate}
\end{document}