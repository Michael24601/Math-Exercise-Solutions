

\documentclass[12pt]{article}

\usepackage[margin=1in]{geometry}

% For using float option H that places figures exatcly where we want them
\usepackage{float}
% makes figure font bold
\usepackage{caption}
\captionsetup[figure]{labelfont=bf}
% For text generation
\usepackage{lipsum}
% For drawing
\usepackage{tikz}
% For smaller or equal sign and not divide sign
\usepackage{amssymb}
% For the diagonal fraction
\usepackage{xfrac}
% For enumerating exercise parts with letters instead of numbers
\usepackage{enumitem}
% For dfrac, which forces the fraction to be in display mode (large) e
% even in math mode (small)
\usepackage{amsmath}
% For degree sign
\usepackage{gensymb}
% For "\mathbb" macro
\usepackage{amsfonts}
\newcommand{\N}{\mathbb{N}}
\newcommand{\Z}{\mathbb{Z}}
\newcommand{\Q}{\mathbb{Q}}
\newcommand{\R}{\mathbb{R}}
\newcommand{\C}{\mathbb{C}}

% overline short italic
\newcommand{\olsi}[1]{\,\overline{{#1}}}
\newcommand{\ang}[1]{\langle #1 \rangle}

\usepackage{indentfirst}
\usetikzlibrary{shapes,positioning,fit,calc}

\usepackage{changepage} % for adjustwidth environment

\title{%
    \Huge Abstract Algebra \\
    \large by \\
    \Large Dummit and Foote \\~\\
    \huge Part 1: Group Theory \\
    \LARGE Chapter 3: Quotient Groups and Homomorphisms \\
    \Large Section 1: Definitions and Examples
}
\date{2024-04-12}
\author{Michael Saba}

\begin{document}
    \pagenumbering{gobble}
    \maketitle
    \newpage
    \pagenumbering{arabic}
 
    This section introduces the concept of a quotient group,
    which, like a subrgoup, allows us to obtain a smaller group
    from another group. 
    Like subgroups, we can use quotient groups of a group to study
    its structure.
    We can often combine information about subgroups and quotient
    groups in order to derive theorems about groups. \\

    \subsection*{Quotient Groups}

    The study of the quotient groups of some group $G$
    is essentially the study of the homomorphisms of $G$. \\
    Suppose that we have a homomorphism $\varphi$
    from $G$ onto another group $H$.
    We know that the fibers of any mapping $\varphi$
    are the sets of elements in $G$ that map to the same element
    in $H$.
    We will call $X_a$ the set of elememts that map to $a$
    where $a \in H$,
    and we say that it is the fiber lying over $a$. \\
    Because the elements being mapped to by $\varphi$
    are in a group $H$,
    the group operation $H$ provides a way of multiplying them.
    This suggests that we can perhaps define an operation
    on the fibers themselves (the sets) that follows
    the group operation of $H$;
    that means that if $a, b \in H$ such that $ab = c$,
    then we could define $X_a \cdot X_b$ to be $X_{ab} = X_{c}$.
    This allows us to create a group out of the fibers of $G$. \\
    We first need to prove that this is indeed a group:
    \begin{itemize}[label=$\diamond$]
        \item 
            The operation is associative because
            the operation is associative in $H$.
        \item
            The identity is the fiber over the identity of $H$,
            $X_1$.
            This is because $X_1 \cdot X_a = X_{1 \cdot a} = X_{a}$
            and $X_a \cdot X_1 = X_{a \cdot 1} = X_{a}$.
        \item
            The inverse of a fiber $X_a$ is the fiber $X_{a^{-1}}$,
            since $X_a \cdot X_{a^{-1}} = X_{aa^{-1}} = X_1$.
        \item
            We know that for any two fibers $X_a$ and $X_b$,
            $X_a \cdot X_b = X_{ab}$.
            In order for the group to be closed under this operation,
            we would need to guarantee that $ab$ has a fiber
            over $\varphi$.
            This is true because, as we know,
            the image of a group $G$ over a homomorphism
            $\varphi: G \to H$
            is closed under the operation of $H$.
            This is because the image of $G$ is itself a group
            (since it is a subgrouo of $H$, 
            as shown in exercise 1.1.6.13,
            and as we will show later in this section).
            Thus the group is closed under its operation.
    \end{itemize}
    The group defined by the fibers of some homomorphism $\varphi$
    from $G$ to $H$
    is known as a \textbf{quotient group} of $G$. \\
    Since the mutliplication of the fibers is defined
    from the multiplication of the elements in $H$,
    it is clear to see that the quotient group is isomorphic
    to the image of $G$ under the homomorphism $\varphi$
    (the elements of $H$ which have non-empty fibers in $G$). \\

    Suppose we have a homomorphism $\varphi: G \to H$.
    The fiber $X_1$ lying over the identity is called
    the kernel of $\varphi$.
    Basically, it is the set
    \[ \{ g \in G \mid \varphi(g) = 1 \} \]
    We can denote it using $\ker \varphi$. \\
    This is related to the idea of the kernel of a group action.
    We can recall that if a group $G$ is acting
    on a a subset $S$ of $G$,
    then the kernel of the action is the set of elements
    in $G$ that satbilize all elements in $S$:
    \[ \{ g \in G \mid g \cdot s = g \text{ for all } s \in S \}  \]
    We can also recall that all group actions
    from a group $G$ onto a set $A$
    are themselves just a homomorphism $\phi: G \to S_A$
    called the permuation representation of the group action
    (since each group element essentially permutes $A$).
    The kernel of the action of $G$ on its subset $S$
    is the set of elements that don't permute $S$,
    and send each of its element to themsleves.
    So if we consider the permutation representation of this action
    $\phi: G \to S_S$,
    then the kernel of the action is the set of elements of $G$
    that $\phi$ maps to the identity permutation
    (the one that doesn't permute $S$),
    which is exactly how we defined the kernel of a homomorphism
    such as $\phi$. \\
    Thus the kernel of group action and the kernel of homomorphism
    are eseentially the same idea. \\

    We will now prove some properties of homomorphisms
    and their fibers.
    Suppose that $\varphi: G \to H$ is a homomorphism:
    \begin{itemize}[label=$\diamond$]
        \item 
            $\varphi(1_G) = \varphi(1_H)$.
            This is true because
            $\varphi(1_G) = \varphi(1_G1_G) = \varphi(1_G)\varphi(1_G)$.
            So by the law of cancellation,
            it must be that $\varphi(1_G) = 1_H$
            (we mutltiply both sides by $\varphi(1_G)^{-1}$).
        \item     
            $\varphi(g^{-1}) = \varphi(g)^{-1}$
            for any $g \in G$.
            This is true because
            $\varphi(1_G) = \varphi(gg^{-1}) 
            = \varphi(g)\varphi(g^{-1}) = 1_H$
            (by the first proposition we proved).
            Multiplying both sides by $\varphi(g)^{-1}$,
            we get $\varphi(g^{-1}) = \varphi(g)^{-1}$.
        \item
            $\varphi(g^n) = \varphi(g)^n$.
            We can prove this using induction:


    \end{itemize}



\end{document}
