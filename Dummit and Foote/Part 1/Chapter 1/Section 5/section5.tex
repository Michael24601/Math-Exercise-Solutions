\documentclass{article}

% For vertical brace rcases
\usepackage{mathtools}
% For positioning figures
\usepackage{float}
% makes figure font bold
\usepackage{caption}
\captionsetup[figure]{labelfont=bf}
% For text generation
\usepackage{lipsum}
% For drawing
\usepackage{tikz}
% For manipulating coordinates
\usetikzlibrary{calc}
% For smaller or equal sign and not divide sign
\usepackage{amssymb}
% For the diagonal fraction
\usepackage{xfrac}
% For enumerating exercise parts with letters instead of numbers
\usepackage{enumitem}
% For dfrac, which forces the fraction to be in display mode (large) e
% even in math mode (small)
\usepackage{amsmath}
% For degree sign
\usepackage{gensymb}
% For "\mathbb" macro
\usepackage{amsfonts}
\newcommand{\N}{\mathbb{N}}
\newcommand{\Z}{\mathbb{Z}}
\newcommand{\Q}{\mathbb{Q}}
\newcommand{\R}{\mathbb{R}}
\newcommand{\C}{\mathbb{C}}
\newcommand{\F}{\mathbb{F}}

% overline short italic
\newcommand{\olsi}[1]{\,\overline{\!{#1}}}

\title{%
    \Huge Abstract Algebra \\
    \large by \\
    \Large Dummit and Foote \\~\\
    \huge Part 1: Group Theory \\
    \LARGE Chapter 1: Introduction to Groups \\
    \Large Section 5: Quaternion Group
}
\date{2023-07-14}
\author{Michael Saba}

\begin{document}
    \pagenumbering{gobble}
    \maketitle
    \newpage
    \pagenumbering{arabic}


    \section*{Exercise 1}
    $|1| = 1$, 
    $|-1| = 2$, 
    $|i| = |j| = |k| = |-i| = |-j| = |-k| = 4$.
    
    
    \section*{Exercise 2}
    The group tables of $S_3$, $D_8$, and $Q_8$ are
    (we apply the row element first in each of the tables):

    \begin{figure}[H]
        \centering

        \[\vbox{\tabskip0.5em\offinterlineskip
        \halign{\strut$#$\hfil\ \tabskip1em\vrule&&$#$\hfil\cr
        ~   & 1 & (1\;2) & (2\;3) & (1\;3) & (1\;2\;3) & (1\;3\;2) \cr
        \noalign{\hrule}\vrule height 12pt width 0pt
        1 & 1 & (1\;2) & (2\;3) & (1\;3) & (1\;2\;3) & (1\;3\;2) \cr
        (1\;2) & (1\;2) & 1 & (1\;2\;3) & (1\;3\;2) & (2\;3) & (1\;3) \cr
        (2\;3) & (2\;3) & (1\;3\;2) & 1 & (1\;2\;3) & (1\;3) & (1\;2) \cr
        (1\;3) & (1\;3) & (1\;2\;3) & (1\;3\;2) & 1 & (1\;2) & (2\;3) \cr
        (1\;2\;3) & (1\;2\;3) & (1\;3) & (1\;2) & (2\;3) & (1\;3\;2) & 1 \cr
        (1\;3\;2) & (1\;3\;2) & (2\;3) & (1\;3) & (1\;2) & 1 & (1\;2\;3) \cr
        }}\]

        \caption{\label{fig:figure1} Group table of $S_3$.}
    \end{figure}

    \begin{figure}[H]
        \centering

        \[\vbox{\tabskip0.5em\offinterlineskip
        \halign{\strut$#$\hfil\ \tabskip1em\vrule&&$#$\hfil\cr
        ~   & 1 & r & r^2 & r^3 & s & sr & sr^2 & sr^3 \cr
        \noalign{\hrule}\vrule height 12pt width 0pt
        1 & 1 & r & r^2 & r^3 & s & sr & sr^2 & sr^3 \cr
        r & r & r^2 & r^3 & 1 & sr & sr^2 & sr^3 & s \cr
        r^2 & r^2 & r^3 & 1 & r & sr^2 & sr^3 & s & sr \cr
        r^3 & r^3 & 1 & r & r^2 & sr^3 & s & sr & sr^2  \cr
        s & s & sr^3 & sr^2 & sr & 1 & r^3 & r^2 & r \cr
        sr & sr & sr^3 & sr^2 & s & r & 1 & r^3 & r^2 \cr
        sr^2 & sr^2 & sr & s & sr^3 & r^2 & r & 1 & r^3 \cr
        sr^3 & sr^3 & sr^2 & sr & s & r^3 & r^2 & r & 1\cr
        }}\]

        \caption{\label{fig:figure1} Group table of $D_8$.}
    \end{figure}

    \begin{figure}[H]
        \centering

        \[\vbox{\tabskip0.5em\offinterlineskip
        \halign{\strut$#$\hfil\ \tabskip1em\vrule&&$#$\hfil\cr
        ~   & 1 & -1 & i & -i & j & -j & k & -k \cr
        \noalign{\hrule}\vrule height 12pt width 0pt
        1 & 1 & -1 & i & -i & j & -j & k & -k \cr
        -1 & -1 & 1 & -i & i & -j & j & -k & k \cr
        i & i & -i & -1 & 1 & -k & k & j & -j \cr
        -i & -i & i & 1 & -1 & k & -k & -j & j \cr
        j & j & -j & k & -k & -1 & 1 & -i & i \cr
        j & -j & j & -k & k & 1 & -1 & i & -i \cr
        k & k & -k & -j & j & i & -i & -1 & 1 \cr
        -k & -k & k & j & -j & -i & i & 1 & -1 \cr
        }}\]

        \caption{\label{fig:figure1} Group table of $Q_8$.}
    \end{figure}


    \section*{Exercise 3}
    $Q_8$ can be generated by $i$ and $j$, where: \\
    $\begin{rcases}
    1 = 1 \hspace{2em} \\
    -1 = i^2 \\
    i = i \\
    -i = i^3 \\
    j = j \\
    -j = j^3 \\
    k = ij \\
    -k = ji \\
    \end{rcases} Q_8 = \langle i, j
    \mid i^2 = j^2, i^4 = 1, i = jij \rangle$ \\

\end{document}