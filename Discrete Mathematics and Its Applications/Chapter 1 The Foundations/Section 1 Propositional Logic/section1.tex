\documentclass{article}


%For 3D plots
\usepackage{blindtext}
\usepackage{pgfplots}
\pgfplotsset{compat=1.9}
% For inline color changes
\usepackage{xcolor}
% For vertical brace rcases
\usepackage{mathtools}
% For positioning figures
\usepackage{float}
% makes figure font bold
\usepackage{caption}
\captionsetup[figure]{labelfont=bf}
% For text generation
\usepackage{lipsum}
% For drawing
\usepackage{tikz}
% For manipulating coordinates
\usetikzlibrary{calc}
% For relative node positioning
\usetikzlibrary{positioning}
% For smaller or equal sign and not divide sign
\usepackage{amssymb}
% For the diagonal fraction
\usepackage{xfrac}
% For enumerating exercise parts with letters instead of numbers
\usepackage{enumitem}
% For dfrac, which forces the fraction to be in display mode (large) e
% even in math mode (small)
\usepackage{amsmath}
% For degree sign
\usepackage{gensymb}
% For quotes
\usepackage{csquotes}
% For "\mathbb" macro
\usepackage{amsfonts}
\newcommand{\N}{\mathbb{N}}
\newcommand{\Z}{\mathbb{Z}}
\newcommand{\Q}{\mathbb{Q}}
\newcommand{\R}{\mathbb{R}}
\newcommand{\C}{\mathbb{C}}
\newcommand{\F}{\mathbb{F}}
\newcommand{\rad}{\text{ rad}}
\newcommand{\lcm}{\text{lcm}}
\newcommand{\Aut}{\text{Aut}}
\newcommand{\seq}{\leqslant}
\newcommand{\beq}{\geqslant}
\newcommand{\sub}{\subseteq}
\newcommand{\prosub}{\subset}

\newcommand{\Ocal}{\mathcal{O}}
\newcommand{\Ccal}{\mathcal{C}}
\newcommand{\Pcal}{\mathcal{P}}
\newcommand{\Scal}{\mathcal{S}}

% overline short italic
\newcommand{\olsi}[1]{\,\overline{\!{#1}}}

\title{%
    \Huge Discrete Mathematics and its Applications \\
    \large by \\
    \Large Kenneth H. Rosen \\~\\
    \huge Chapter 1: The Foundations \\
    \LARGE Section 1: Propositional Logic \\
}
\date{2023-09-24}
\author{Michael Saba}

\begin{document}
    \pagenumbering{gobble}
    \maketitle
    \newpage
    \pagenumbering{arabic}


    \section*{Exercise 1}
    A proposition has to be true or false,
    so that means:
    \begin{enumerate}[label=\textbf{\alph*.}]
        \item
            \enquote{Boston is the capital of Massachusetts} 
            is a proposition (true).
        \item
            \enquote{Miami is the capital of Florida} 
            is a proposition (false).
        \item 
            $2 + 3 = 5$ is a proposition (true).
        \item 
            $5 + 7 = 10$ is a proposition (false).
        \item 
            $x + 2 = 11$ is not a proposition 
            (can be true or false depending on $x$).
        \item
            \enquote{Answer this question} 
            doesn't even make a claim,
            so not a proposition.
    \end{enumerate}


    \section*{Exercise 2}
    Same as exercise 1.1.1, so it can be skipped.


    \section*{Exercise 3}
    \begin{enumerate}[label=\textbf{\alph*.}]
        \item
            Linda is older or has the same age as Sanjay.
        \item
            Mei makes less or as much money as Isabella. 
        \item 
            Moshe is shorter or has the same height as Monica.
        \item 
            Abby is poorer or just as rich as Ricardo.
    \end{enumerate}


    \section*{Exercise 4}
    Same as exercise 1.1.3, so it can be skipped.


    \section*{Exercise 5}
    Same as exercise 1.1.3, so it can be skipped.


    \section*{Exercise 6}
    Same as exercise 1.1.3, so it can be skipped.


    \section*{Exercise 7}
    Same as exercise 1.1.3, so it can be skipped.


    \section*{Exercise 8}
    \begin{enumerate}[label=\textbf{\alph*.}]
        \item
            True.
        \item
            True.
        \item 
            False.
        \item 
            False.
        \item
            False.
    \end{enumerate}


    \section*{Exercise 9}
    Same as exercise 1.1.8, so it can be skipped.


    \section*{Exercise 10}
    We have 
    \[ p \text{: I bought a lottery ticket this week.} \]
    \[ q \text{: I won the million dollar jackpot.} \]
    So
    \begin{enumerate}[label=\textbf{\alph*.}]
        \item
            $\neg p$: I did not buy a lottery ticket this week.
        \item
            $p \lor q$: I bought a lottery ticket this week
            or won the million dollar jackpot.
        \item 
            $p \implies q$: If I bought a lottery ticket this week,
            then I won the million dollar jackpot.
        \item 
            $p \land q$: I bought a lottery ticket this week
            and I won the million dollar jackpot.
        \item
            $p \iff q$: I bought a lottery ticket this week if and only if
            I won the million dollar jackpot.
        \item
            $\neg p \implies \neg q$:
            If I did not buy a lottery ticket this week,
            then I did not win the million dollar jackpot.
        \item
            $\neg p \land \neg q$:
            I did not buy a lottery ticket this week
            and I did not win the million dollar jackpot.
        \item 
            $\neg p \lor (p \land q)$:
            I did not buy a lottery ticket this week
            or I bought a lottery ticket this week
            and I wont the million dollar jackpot.
    \end{enumerate}


    \section*{Exercise 11}
    Same as exercise 1.1.10, so it can be skipped.

    
    \section*{Exercise 12}
    Same as exercise 1.1.10, so it can be skipped.


    \section*{Exercise 13}
    We have 
    \[ p \text{: It is below freezing.} \]
    \[ q \text{: It is snowing.} \]
    So
    \begin{enumerate}[label=\textbf{\alph*.}]
        \item
            It is below freezing and snowing:
            $p \land q$.
        \item
            It is below freezing but not snowing:
            $p \land \neg q$.
        \item 
            It is not below freezing and it is not snowing:
            $\neg p \land \neg q$.
        \item 
            It is either snowing or below freezing (or both):
            $p \lor q$.
        \item
            If it is below freezing, it is also snowing:
            $p \implies q$.
        \item
            Either it is below freezing or it is snowing, but it is
            not snowing if it is below freezing:
            $(p \lor q) \land (p \implies \neg q)$.
        \item
            That it is below freezing is necessary and sufficient
            for it to be snowing:
            $p \iff q$.
    \end{enumerate}


    \section*{Exercise 14}
    Same as exercise 1.1.13, so it can be skipped.


    \section*{Exercise 15}
    Same as exercise 1.1.13, so it can be skipped.

    \section*{Exercise 16}
    Same as exercise 1.1.13, so it can be skipped.

    
    \section*{Exercise 17}
    Same as exercise 1.1.13, so it can be skipped.

\end{document}
