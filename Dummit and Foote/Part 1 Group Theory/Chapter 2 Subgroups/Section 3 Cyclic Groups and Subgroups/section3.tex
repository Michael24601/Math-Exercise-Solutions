\documentclass{article}

% For vertical brace rcases
\usepackage{mathtools}
% For positioning figures
\usepackage{float}
% makes figure font bold
\usepackage{caption}
\captionsetup[figure]{labelfont=bf}
% For text generation
\usepackage{lipsum}
% For drawing
\usepackage{tikz}
% For manipulating coordinates
\usetikzlibrary{calc}
% For smaller or equal sign and not divide sign
\usepackage{amssymb}
% For the diagonal fraction
\usepackage{xfrac}
% For enumerating exercise parts with letters instead of numbers
\usepackage{enumitem}
% For dfrac, which forces the fraction to be in display mode (large) e
% even in math mode (small)
\usepackage{amsmath}
% For degree sign
\usepackage{gensymb}
% For "\mathbb" macro
\usepackage{amsfonts}
\newcommand{\N}{\mathbb{N}}
\newcommand{\Z}{\mathbb{Z}}
\newcommand{\Q}{\mathbb{Q}}
\newcommand{\R}{\mathbb{R}}
\newcommand{\C}{\mathbb{C}}
\newcommand{\F}{\mathbb{F}}
\newcommand{\rad}{\text{ rad}}
\newcommand{\lcm}{\text{lcm}}

% overline short italic
\newcommand{\olsi}[1]{\,\overline{\!{#1}}}

\title{%
    \Huge Abstract Algebra \\
    \large by \\
    \Large Dummit and Foote \\~\\
    \huge Part 1: Group Theory \\
    \LARGE Chapter 2: Subgroups \\
    \Large Section 3: Cyclic Groups and Cyclic Subgroups
}
\date{2023-07-14}
\author{Michael Saba}

\begin{document}
    \pagenumbering{gobble}
    \maketitle
    \newpage
    \pagenumbering{arabic}


    \section*{Exercise 1 $***$}
    For $Z_{45} = \langle x \rangle$,
    we know that $\forall n \in \Z^+$,
    if $n \mid |Z_{45}|$,
    then $\langle x^n \rangle$ is a unique subgroup of $Z_{45}$.
    Moreover, we know that for any $m \in \Z^+$,
    $\langle x^m \rangle = \langle x^{\gcd(m, |Z_{45}|)} \rangle$.
    So the unique cyclic subgroups generated by the divisiors of $|Z_45|$
    are the only subgroups of $Z_{45}$.
    The divisors of $|Z_{45}| = 45$ are $1, 3, 5, 9, 15, 45$,
    so we have 6 subgroups $\langle x \rangle$, 
    $\langle x^3 \rangle$, $\langle x^5 \rangle$,
    $\langle x^9 \rangle$, $\langle x^{15} \rangle$,
    and $\langle x^{45} \rangle = \langle x^0 \rangle = \{1\}$. \\
    We know that for each subgroup
    $\langle x^m \rangle$ of $\langle x \rangle$,
    \[ |\langle x^m \rangle|
    = \dfrac{|\langle x \rangle|}{\gcd(|\langle x \rangle|, m)} \]
    But since we assumed that $m \mid |\langle x \rangle|$,
    then 
    \[ |\langle x^m \rangle|
    = \dfrac{|\langle x \rangle|}{m} \]
    In order for a cyclic subgroup of $\langle x \rangle$
    to be the subgroup of another cyclic subgroup of $\langle x \rangle$,
    the same property applied above must be met.
    So for a a subgroup $\langle x^p \rangle$ with order 
    $\sfrac{|\langle x \rangle|}{p}$,
    $\langle (x^p)^q \rangle$ is a unique subgroup of this subgroup
    as long as $q \in \Z$ and $q \mid |\langle x^p \rangle|
    = \sfrac{|\langle x \rangle|}{p}$.
    In our case, we need to find all integers $q$
    that divide the order of each subgroup,
    such that the element $x^{pq}$ generates one of the subgroups
    of $\langle x \rangle$
    (since these are all the unique subgroups, and any other cyclic
    subgroup is just the same as one of them).
    So: \\
    For $n = 1$, $\langle x \rangle
    \geqslant \langle x^3 \rangle 
    \geqslant \langle x^9 \rangle
    \geqslant \langle x^{15} \rangle
    \geqslant \langle x^{45} \rangle = \langle x^0 \rangle = \{1\}$ \\
    For $n = 3$, $\langle x^3 \rangle 
    \geqslant \langle x^9 \rangle
    \geqslant \langle x^{15} \rangle
    \geqslant \langle x^{45} \rangle = \langle x^0 \rangle = \{1\}$ \\
    For $n = 9$, $\langle x^9 \rangle
    \geqslant \langle x^{45} \rangle = \langle x^0 \rangle = \{1\}$ \\
    For $n = 15$, $\langle x^{15} \rangle
    \geqslant \langle x^{45} \rangle = \langle x^0 \rangle = \{1\}$ \\
    For $n = 45$, $\langle x^{45} \rangle = \langle x^0 \rangle = \{1\}$ \\

    
    \section*{Exercise 2}
    Proof that for $x \in G$, if $|x| = |G| < \infty$,
    then $G = \langle x \rangle:$ \\
    If $|x| = n$,
    then by exercise 1.1.1.32,
    $x^0, x^1, x^2 \dots x^{n-1}$ are all unique.
    Since $|G| = |x| = n$,
    there are $n$ elements in $G$,
    and the $n$ unique elements generated by $x$ that we listed
    are all in $G$ as it is closed under the group operation.
    So those elements are the only elements in $G$,
    and since they are generated by $x$ alone, 
    $G = \langle x \rangle$. \\
    Proof that this isn't necessarily the case if $|x| = |G| < \infty$: \\
    We can prove this by giving a counterexample.
    For instance, take $G = (\Z, +)$ and $x = 2$.
    We know all elements in the additive group of integers have
    infinite order,
    same as the group itself.
    However, all elements that can be generated by 2 are of the form $2n$,
    which is always even.
    So not all of $\Z$ can be generated by 2,
    hence $\Z \neq \langle 2 \rangle$.


    \section*{Exercise 3}
    The number of generators of of the group $\Z/48\Z$
    is $\varphi(|\Z/48\Z|)$ where $\varphi$
    is \textit{Euler's totient function}.
    This is because $\varphi(a) \; \forall a \in \Z^+$
    is by defintion the number of numbers between 1 and $a$
    that are relatively prime to $a$.
    We know that the only unique elements in $\Z/48\Z$
    are $\olsi{0}, \olsi{1}, \olsi{2} \dots \olsi{47}$
    (any other element is the same as one of these since $|\Z/48\Z| = 48$).
    We also know that for any cyclic groups,
    $\langle x^m \rangle = \langle x^{\gcd(m, |\langle x \rangle|)} \rangle$,
    So as $\Z/48\Z = \langle \olsi{1} \rangle$,
    $\langle m \cdot \olsi{1} \rangle
    = \langle \gcd(m, |\langle \olsi{1} \rangle|) \cdot x \rangle$.
    So in order to have $\langle \olsi{m} \rangle = \Z/48\Z$
    where $\olsi{m} \in \Z/48\Z$, $1 \leqslant m < 48$,
    it would need to satisfy the equation
    which means that $\gcd(1, |\langle x \rangle|)$
    \[ \Z/48\Z = \langle \olsi{1} \rangle
    =  \langle 1 \cdot \olsi{1} \rangle
    =  \langle \olsi{m} \rangle
    =  \langle m \cdot \olsi{1} \rangle
    =  \langle \gcd(m, |\langle \olsi{1} \rangle|) \cdot \olsi{1} \rangle
    =  \langle \gcd(m, 48) \cdot \olsi{1} \rangle \]
    so $\gcd(m, 48) = 1$,
    completing the proof. \\
    So $\langle \olsi{1} \rangle
    = \langle \olsi{5} \rangle = \langle \olsi{7} \rangle
    = \langle \olsi{11} \rangle = \langle \olsi{13} \rangle
    = \langle \olsi{17} \rangle = \langle \olsi{19} \rangle
    = \langle \olsi{23} \rangle = \langle \olsi{25} \rangle
    = \langle \olsi{29} \rangle = \langle \olsi{31} \rangle
    = \langle \olsi{35} \rangle = \langle \olsi{37} \rangle
    = \langle \olsi{41} \rangle = \langle \olsi{43} \rangle
    = \langle \olsi{47} \rangle$.


    \section*{Exercise 4}
    As per the proof in exercise 1.2.3.3,
    the generators of $\Z/202\Z$ are the residue classes of 202
    that are relatively prime to 202
    which can be factored into $202 = 101 \times 2$.
    Since both are primes, all odd integers' residue classes
    from 1 to 199 excluding 101 (and including 202) are generators
    of $\Z/202\Z$.


    \section*{Exercise 5}
    One property of $\varphi$, Euler's totient function,
    is that $\varphi(a)
    = \varphi(p_1^{r_1}) \cdot \varphi(p_2^{r_2}) \dots \varphi(p_n^{r_n})$
    where $p_1, p_2 \dots p_n$ are unique prime factors of $a$
    as per \textit{the Fundamental theorem of Arithmetic}.
    So for $49000 = 2^3 \cdot 5^3 \cdot 7^2$,
    the number of generators of $\Z/49\Z$
    is $\varphi(49000) = \varphi(2^3) \cdot \varphi(5^3) \dots \varphi(7^2)
    = 4 \cdot 100 \cdot 42 = 16800$.


    \section*{Exercise 6}
    We know that $\Z/48\Z = \{\olsi{0}, \olsi{1} \dots \olsi{47}\}$. \\
    Following the same method we used in exercise 1.2.3.1
    to find all subgroups of $Z_{45}$ and their inclusions,
    we get that: \\
    $\langle \olsi{1} \rangle
    = \langle \olsi{5} \rangle = \langle \olsi{7} \rangle
    = \langle \olsi{11} \rangle = \langle \olsi{13} \rangle
    = \langle \olsi{17} \rangle = \langle \olsi{19} \rangle
    = \langle \olsi{23} \rangle = \langle \olsi{25} \rangle
    = \langle \olsi{29} \rangle = \langle \olsi{31} \rangle
    = \langle \olsi{35} \rangle = \langle \olsi{37} \rangle
    = \langle \olsi{41} \rangle = \langle \olsi{43} \rangle
    = \langle \olsi{47} \rangle
    = \{\olsi{0}, \olsi{1} \dots \olsi{47}\}$. \\
    $\langle \olsi{1} \rangle \geqslant
    \langle \olsi{2} \rangle,
    \langle \olsi{3} \rangle,
    \langle \olsi{4} \rangle,
    \langle \olsi{6} \rangle,
    \langle \olsi{8} \rangle,
    \langle \olsi{12} \rangle,
    \langle \olsi{16} \rangle,
    \langle \olsi{24} \rangle,
    \langle \olsi{0} \rangle$. \\
    $\langle \olsi{2} \rangle
    = \langle \olsi{10} \rangle = \langle \olsi{14} \rangle
    = \langle \olsi{22} \rangle = \langle \olsi{26} \rangle
    = \langle \olsi{34} \rangle = \langle \olsi{38} \rangle
    = \langle \olsi{46} \rangle 
    = \{\olsi{0}, \olsi{2}, \olsi{4} \dots \olsi{46}\}$. \\
    $\langle \olsi{2} \rangle \geqslant
    \langle \olsi{2} \rangle,
    \langle \olsi{4} \rangle,
    \langle \olsi{6} \rangle,
    \langle \olsi{8} \rangle,
    \langle \olsi{12} \rangle,
    \langle \olsi{16} \rangle,
    \langle \olsi{24} \rangle,
    \langle \olsi{0} \rangle$. \\
    $\langle \olsi{3} \rangle
    = \langle \olsi{9} \rangle = \langle \olsi{15} \rangle
    = \langle \olsi{21} \rangle = \langle \olsi{27} \rangle
    = \langle \olsi{33} \rangle = \langle \olsi{39} \rangle
    = \langle \olsi{45} \rangle 
    = \{\olsi{0}, \olsi{3}, \olsi{6} \dots \olsi{45}\}$. \\
    $\langle \olsi{3} \rangle \geqslant
    \langle \olsi{3} \rangle,
    \langle \olsi{6} \rangle,
    \langle \olsi{12} \rangle,
    \langle \olsi{24} \rangle,
    \langle \olsi{0} \rangle$. \\
    $\langle \olsi{4} \rangle
    = \langle \olsi{4} \rangle = \langle \olsi{20} \rangle
    = \langle \olsi{28} \rangle = \langle \olsi{44} \rangle
    = \{\olsi{0}, \olsi{4}, \olsi{8} \dots \olsi{44}\}$. \\
    $\langle \olsi{4} \rangle \geqslant
    \langle \olsi{4} \rangle,
    \langle \olsi{8} \rangle,
    \langle \olsi{12} \rangle,
    \langle \olsi{16} \rangle,
    \langle \olsi{24} \rangle,
    \langle \olsi{0} \rangle$. \\
    $\langle \olsi{6} \rangle
    = \langle \olsi{18} \rangle = \langle \olsi{30} \rangle
    = \langle \olsi{42} \rangle
    = \{\olsi{0}, \olsi{6}, \olsi{12} \dots \olsi{42}\}$. \\
    $\langle \olsi{6} \rangle \geqslant
    \langle \olsi{6} \rangle,
    \langle \olsi{12} \rangle,
    \langle \olsi{24} \rangle,
    \langle \olsi{0} \rangle$. \\
    $\langle \olsi{8} \rangle
    = \langle \olsi{40} \rangle
    = \{\olsi{0}, \olsi{8}, \olsi{16} \dots \olsi{40}\}$. \\
    $\langle \olsi{8} \rangle \geqslant
    \langle \olsi{8} \rangle,
    \langle \olsi{16} \rangle,
    \langle \olsi{24} \rangle,
    \langle \olsi{0} \rangle$. \\
    $\langle \olsi{12} \rangle
    = \langle \olsi{36} \rangle
    = \{\olsi{0}, \olsi{12}, \olsi{24}, \olsi{36}\}$. \\
    $\langle \olsi{12} \rangle \geqslant
    \langle \olsi{12} \rangle,
    \langle \olsi{24} \rangle,
    \langle \olsi{0} \rangle$. \\
    $\langle \olsi{16} \rangle
    = \langle \olsi{32} \rangle
    = \{\olsi{0}, \olsi{16}, \olsi{32}\}$. \\
    $\langle \olsi{16} \rangle \geqslant
    \langle \olsi{16} \rangle,
    \langle \olsi{0} \rangle$. \\
    $\langle \olsi{24} \rangle
    = \{\olsi{0}, \olsi{24}\}$. \\
    $\langle \olsi{24} \rangle \geqslant
    \langle \olsi{24} \rangle,
    \langle \olsi{0} \rangle$. \\
    $\langle \olsi{0} \rangle
    = \{\olsi{0}\}$. \\
    $\langle \olsi{0} \rangle \geqslant
    \langle \olsi{0} \rangle$. \\
    

    \section*{Exercise 7}
    For $Z_{48} = \langle x \rangle$,
    we have an isomorphism $\phi: \Z/48\Z \to Z_{48}$
    defined by $\phi(\olsi{1}) = x$. \\
    This isomorphism clearly maps $\phi(\olsi{a}) = x^a$
    when $0 \leqslant a < 48$.
    We can prove this by induction
    (starting with $n = 2$, not 1, as
    $\phi(\olsi{1}) = x$ can't on its own tell us much): \\
    \textbf{Basis step:}
    We know that for $n = 2$,
    because $\phi$ is a homomorphism,
    $\phi(\olsi{2}) = \phi(\olsi{1} + \olsi{1})
    = \phi(\olsi{1}) \cdot \phi(\olsi{1}) = x \cdot x = x^2$. \\
    \textbf{Inductive hypothesis:}
    Assume that for $n = k$, $\phi(\olsi{k}) = x^k$ \\ 
    \textbf{Inductive step:}
    For $n = k + 1$, because $\phi$ is a homomorphism,
    $\phi(\olsi{k + 1}) = \phi(\olsi{k} + \olsi{1})
    = \phi(\olsi{k}) \cdot \phi(\olsi{1}) = x^k \cdot x = x^{k+1}$. \\
    So as $\phi(\olsi{a}) = x^a$,
    we can map the subgroups of $\Z/48\Z$ found in exercise 1.2.3.6
    using $\phi$ to get the subgroups of $Z_48$: \\
    $\langle x \rangle \geqslant
    \langle x \rangle,
    \langle x^3 \rangle,
    \langle x^4 \rangle,
    \langle x^6 \rangle,
    \langle x^8 \rangle,
    \langle x^{12} \rangle,
    \langle x^{16} \rangle,
    \langle x^{24} \rangle,
    \langle x^0 \rangle = \langle 1 \rangle$.
    

    \section*{Exercise 8 $***$}
    For $Z_{48} = \langle x \rangle$,
    consider the homomorphism $\varphi_a: \Z/48\Z \to Z_{48}$
    defined by $\varphi_a(\olsi{1}) = x^a$.
    We need to find the values of $a$ for which the map extends to
    an isomorphism. \\
    Since the function only requires that $\varphi_a(\olsi{1}) = x^a$,
    we can add a new constraint to $\varphi_a$ which says that 
    $\varphi_a(\olsi{n}) = (x^a)^n$ for any $\olsi{n} \in \Z/48\Z$.
    Since the given condition that $\varphi_a(\olsi{1}) = x$
    still holds for $n = 1$, this is perfectly allowed. \\
    First, we need to show that the map is well defined.
    That means that for $\olsi{m} = \olsi{n}$,
    we have $\varphi_a(\olsi{m}) = \varphi_a(\olsi{n})$.
    Note that we know that $\olsi{m} = \olsi{n}$ in $\Z/48\Z$
    means that $m \equiv n \mod 48$ by definition.
    Now, $\varphi_a(\olsi{m}) = \varphi_a(\olsi{n})$
    implies that $(x^a)^m = (x^a)^n$.
    From exercise 1.1.1.35, and the use of the division
    algorithm, we know that for $x^k = 1$, $x^p = x^{(p \mod k)}$
    (applies whether or not $k$ is the order of $x$).
    So in $Z_{48}$, we have $(x^{48})^a = 1^a = 1$,
    which in turn means that $(x^a)^{48} = 1$.
    And since we have $m \equiv n \mod 48$, that must mean
    that
    \[ x^m = x^{(m \mod 48)} = x^{(n \mod 48)} = x^n \]
    So $\varphi_a$ is well defined. \\
    Now, we can see that
    \[ \varphi_a(\olsi{m} + \olsi{n}) = (x^a)^{m + n}
    = (x^a)^m(x^a)^n
    = \varphi_a(\olsi{m})\phi(\olsi{n})  \]
    which makes $\varphi_a$ a homomorphism. \\
    We know that $|\Z/48\Z| = |Z_{48}| = 48$.
    So in order for the homomorphism to be an isomorphism,
    we first need to show that it is a bijection,
    so we need $\varphi_a$ to either be injective or surjective.
    We will work with the former. \\
    If $\varphi_a$ is injective,
    then for $\olsi{b} \neq \olsi{c}$,
    $\varphi_a(b) \neq \varphi_a(c)$,
    so $(x^a)^b \neq (x^a)^c$,
    which means that $x^{ab} \neq x^{ac}$.
    From exercise 1.1.1.35, and the use of the division
    algorithm, we know that for $|x| = n$, $x^m = x^{(m \mod n)}$.
    So $x^{ab} \neq x^{ac}$ implies that $ab \not\equiv ac \mod |x|$
    where $|x| = 48$ since the group $Z_{48}$ is cyclic.
    So $48 \nmid (ab - ac)$,
    which means that $48 \nmid a(b - c)$.
    We know that $48 \nmid (b - c)$
    as we assumed that $\olsi{b} \neq \olsi{c}$,
    which means that $b \not\equiv c \mod 48$.
    However, while we do know that $48 \nmid (b - c)$,
    we also know that $b - c$ can have many values that
    share factors with 48.
    So in order to ensure that the equations holds for all these values,
    there must be no factors of 48 in $a$.
    So $\gcd(a, 48) = 1$. \\
    This makes $\varphi_a$ an injection,
    which means it is a surjection,
    which in turn makes it an isomorphism. \\
    This result is also unsurprising.
    We know that $\olsi{1}$ is a generator of $\langle x \rangle$,
    and we know that an isomorphism maps the generators of the
    two groups to each other.
    By definition, we know that $\langle x^a \rangle$
    is a generators with order $48$
    (same as $\olsi{1}$, another thing isomorphisms preserve)
    if and only if $\gcd(a, 48) = 1$.


    \section*{Exercise 9 $***$}
    For $Z_{36} = \langle x \rangle$,
    consider map $\psi_a: \Z/48\Z \to Z_{36}$
    defined by $\psi_a(\olsi{1}) = x^a$.
    We need to find the values of $a$ for which the map is a well
    defined homomorphism. \\
    Since the function only requires that $\psi_a(\olsi{1}) = x^a$,
    we can add a new constraint to $\varphi_a$ which says that 
    $\psi_a(\olsi{n}) = (x^a)^n$ for any $\olsi{n} \in \Z/48\Z$.
    Since the given condition that $\psi_a(\olsi{1}) = x$
    still holds for $n = 1$, this is perfectly allowed. \\
    First, we need to show that the map is well defined.
    That means that for $\olsi{m} = \olsi{n}$,
    we have $\psi_a(\olsi{m}) = \psi_a(\olsi{n})$.
    Note that we know that $\olsi{m} = \olsi{n}$ in $\Z/48\Z$
    means that $m \equiv n \mod 48$ by definition.
    Now, $\psi_a(\olsi{m}) = \psi_a(\olsi{n})$
    implies that $(x^a)^m = (x^a)^n$,
    which in turn implies that $x^{am} = x^{an}$.
    From exercise 1.1.1.35, and the use of the division
    algorithm, we kmow that for $|x| = k$, $x^p = x^{(p \mod k)}$.
    So $x^{am} = x^{an}$ implies that $ab \equiv ac \mod |x|$
    where $|x| = 36$ since the group $Z_{36}$ is cyclic.
    So $36 \mid (am - an)$,
    which means that $36 \mid a(m - n)$.
    We know that $48 \mid (m - n)$,
    so $m - n = 48d$ for some $d \in \Z$. \\
    Hence $36 \mid 48ad$,
    which means that $3 \mid 4ad$.
    We know 3 is prime, so by \textit{Euclid's lemma},
    3 must divide at least one of the factors.
    We know that $3 \nmid 4$,
    and we can't fix the value of $d$ since $m$ and $n$
    can be any integers,
    so we're left with $3 \mid a$,
    which means that $\psi_a$ is well defined when
    $a = 3r$ for some $r \in \Z$. \\
    We can then follow this by a proof that this forms a homomorphism:
    \[ \psi_a(\olsi{m} + \olsi{n}) = (x^a)^{m + n}
    = (x^a)^m(x^a)^n
    = \psi_a(\olsi{m})\psi_a(\olsi{n})  \]
    which makes $\psi_a$ a well defined homomorphism. \\
    Proof that $\psi_a$ is never surjective:
    $\psi_a$ is not surjective because it doesn't map any element
    to $x$ in $Z_{36}$.
    Assume, by contradiction, that for some $\olsi{b} \in \Z/48\Z$,
    $\psi_a(\olsi{b}) = x$.
    This implies that $x^{ab} = x$,
    which means that $x^{ab}x^{-1} = 1$,
    so $x^{ab - 1} = 1$,
    and finally, $x^{3rb - 1} = 1$.
    We know from a proposition in the book that
    if $x^m = 1$, then $|x| \mid m$.
    Since $|x| = 36$ in $Z_{36}$,
    that means that $36 \mid 3rb - 1$,
    which means that $3rb - 1 = 36s$ for some $s \in \Z$.
    So $1 = (3rb - 36s)$,
    which means that $1 = 3(rb - 12s)$,
    implying that 1 is a multiple of 3 (or is equal to 0),
    which is a contradiction.
    So $x$ is not the image of any elements in $\Z/48\Z$,
    meaning that $\psi_a$ can't be surjective.


    \section*{Exercise 10}
    Knowing that $\olsi{30} = 30 \cdot \olsi{1}$,
    the order of $\olsi{30}$ in $\Z/54\Z$ is
    \[ \dfrac{|\Z/54\Z|}{\gcd(|\Z/54\Z|, |\olsi{30}|)}
    = \dfrac{54}{\gcd(54, 30)} 
    = \dfrac{54}{6}
    = 9 \]
    We know that $\langle \olsi{30} \rangle
    = \{ \olsi{0}, \olsi{30}, \olsi{6}, \olsi{36}, \olsi{12}, 
    \olsi{42}, \olsi{18}, \olsi{48}, \olsi{24}\}$,
    where the order of each element is: $|\olsi{0}| = 1$,
    $|\olsi{30}| = 9$, $|\olsi{6}| = 9$, $|\olsi{36}| = 3$
    $|\olsi{12}| = 9$, $|\olsi{42}| = 9$, $|\olsi{18}| = 3$,
    $|\olsi{48}| = 9$, $|\olsi{24}| = 9$.


    \section*{Exercise 11}
    The cyclic subgroups of $D_8$ are (we check for each element): \\
    $\langle 1 \rangle = \{1\}$ \\
    $\langle r \rangle = \langle r^3 \rangle = \{1, r, r^2, r^3\}$ \\
    $\langle r^2 \rangle = \{1, r^2\}$ \\
    $\langle s \rangle = \{1, s\}$ \\
    $\langle sr \rangle = \{1, sr\}$ \\
    $\langle sr^2 \rangle = \{1, sr^2\}$ \\
    $\langle sr^3 \rangle = \{1, sr^3\}$ \\
    A proper subgroup of a group $G$ is a subgroup of $G$ that isn't $G$.
    We can denote it by $H < G$.
    An example of a proper subgroup of $D_8$
    that isn't cyclic is the one that features in exercise 1.2.2.5
    which the exercise tells us is a subgroup of $D_8$:
    $\{1, s, r^2, sr^2\}$.


    \section*{Exercise 12}
    Proof these groups are not cyclic: 
    \begin{enumerate}[label=\textbf{\alph*.}]
        \item 
            For $Z_2 \times Z_2$,
            where $Z_2 = \langle x \rangle$: \\
            We have $\langle x \rangle = \{1, x\}$,
            but none of the combinations in a single tuple
            can generate the whole group,
            which is $\{(1, 1), (1, x), (x, 1), (x, x)\}$.
            This is because for any $n \in \Z$,
            $(1, 1)^n = (1, 1)$,
            $(1, x)^n$ and $(x, 1)^n = (1, x)$ or $(x, 1)$,
            and $(x, x)^n = (1, 1)$ or $(x, x)$.
            So $Z_2 \times Z_2$ isn't generated by one element,
            and is thus not cyclic.
        \item
            For $Z_2 \times \Z$,
            where $Z_2 = \langle x \rangle$
            and $\Z$ is the additive group: \\
            We have $\langle x \rangle = \{1, x\}$
            and $\Z = \langle 1 \rangle$.
            So the generator, for some $a \in \Z$,
            will be of the form $(x, a)$
            as the identities of both groups can't generate
            any elements except themsleves.
            However, this means that we can't generate $(x, 2a)$.
            Because $a \neq 0$, so $|a| = \infty$ in $\Z$,
            which means that only $2 \cdot a$ gives $2a$.
            But $(x, a)$ repeated twice is $(x, a)^2 = (1, 2a)$,
            and as every other power that gives $2a$ is a multiple of 2,
            the first element in the ordered pair will always be 1. 
            So $Z_2 \times \Z$ isn't generated by one element,
            and is thus not cyclic.
        \item
            For $\Z \times \Z$,
            where $\Z$ is the additive group: \\
            We have $\Z = \langle 1 \rangle$.
            So the generator, for some $a, b \in \Z$,
            will be of the form $(a, b)$,
            where neither are 0 
            since the identities of both groups can't generate any elements
            except themsleves.
            However, this means we can never generate $(na, mb)$
            if $n \neq m$.
            This is because $a, b \neq 0$,
            so $|a| = |b| = \infty$,
            which means that $n \cdot a$ and $m \cdot b$ is the only
            way to generate $na$ and $mb$.
            However, we can't both have $(a, b)^n$ and $(a, b)^m$
            since $n \neq m$.
            We also can't do something like take a multiple of $n$ and $m$,
            since both $a$ and $b$ have infinite order,
            so $mn \cdot a \neq na$ and $nm \cdot b \neq mb$.
            So $\Z \times \Z$ isn't generated by one element,
            and is thus not cyclic.
    \end{enumerate}


    \section*{Exercise 13}
    \begin{enumerate}[label=\textbf{\alph*.}]
        \item 
            Proof that $\Z \times Z_2 \ncong \Z$
            where $\Z$ is the additive group of integers
            and $Z_2 = \langle x \rangle$: \\
            We know form exercise 1.1.6.2 that isomorphic
            groups have the same number of elements with each order.
            However, $\Z$ contains $(0, x)$, which has order 2
            since $(0, x) \neq (0, 1)$
            (the identity according to exercise 1.1.1.28)
            but $(0, x)^2 = (2 \cdot 0, x^2) = (0, 1)$.
            On the other hand, $\Z$ is an infinite and cyclic
            group, which means that only 0, the identity,
            has a finite order, and it's 1.
            So as $\Z$ contains no elements of order 2,
            it can't be isomorphic to $\Z \times Z_2$.
        \item
            Proof that $\Q \times Z_2 \ncong \Q$
            where $\Q$ is the additive group of integers
            and $Z_2 = \langle x \rangle$: \\
            The same proof that we used in part a applies
            since $\Q$ is also an infinite cyclic group. \\
            We know form exercise 1.1.6.2 that isomorphic
            groups have the same number of elements with each order.
            However, $\Q$ contains $(\sfrac{0}{1}, x)$,
            which has order 2
            since $(0, x) \neq (0, 1)$
            (the identity according to exercise 1.1.1.28)
            but $(0, x)^2 = (2 \cdot 0, x^2) = (0, 1)$.
            On the other hand, $\Q$ is an infinite and cyclic
            group, which means that only 0, the identity,
            has a finite order, and it's 1.
            So as $\Q$ contains no elements of order 2,
            it can't be isomorphic to $\Q \times Z_2$,
    \end{enumerate}


    \section*{Exercise 14}
    We know from exercise 1.1.1.35, and the use of the division
    algorithm, that for $|x| = n$, $x^m = x^{(m \mod n)}$.
    So for $\sigma = (1\;2\;3\;4\;5\;6\;7\;8\;9\;10\;11\;12)$.
    Since $\sigma$ is a 10-tuple, $|\sigma| = 10$,
    so $\sigma^a = \sigma^{(a \mod 10)}$.
    We also know every single power of $\sigma$ from 1 to 12,
    which we calculated in exercise 1.1.3.9.
    So we have: \\
    $\sigma^{13} = \sigma^1 = (1\;2\;3\;4\;5\;6\;7\;8\;9\;10\;11\;12)$ \\
    $\sigma^{65} = \sigma^5 = (1\;7)(2\;8)(3\;9)(4\;10)(5\;11)(6\;12)$ \\
    $\sigma^{626} = \sigma^2 = (1\;3\;5\;7\;9\;11)(2\;4\;6\;8\;10\;12)$ \\
    $\sigma^{1195} = \sigma^7 = (1\;8\;3\;10\;5\;12\;7\;2\;9\;4\;11\;6)$ \\
    $\sigma^{-6} = \sigma^6 = (1\;6\;11\;4\;9\;2\;7\;12\;5\;10\;3\;8)$ \\
    $\sigma^{-81} = \sigma^3 = (1\;4\;7\;10)(2\;5\;8\;11)(3\;6\;9\;12)$ \\
    $\sigma^{-570} = \sigma^6 = (1\;6\;11\;4\;9\;2\;7\;12\;5\;10\;3\;8)$ \\
    $\sigma^{-1211} = \sigma^1 = (1\;2\;3\;4\;5\;6\;7\;8\;9\;10\;11\;12)$


    \section*{Exercise 15}
    Proof that $\Q \times \Q$ is not cyclic
    where $\Q$ is the additive group of rationals: \\
    We know that all subgroups of a cyclic group are cyclic.
    We also know from exercise 1.2.3.12 that $\Z \times \Z$
    is not cyclic.
    Since all integers $x, y \in \Z$ are also in $\Q$,
    then all elements $(x, y) \in \Q \times \Q$,
    so $\Z \times \Z \subseteq \Q \times \Q$.
    Moreover, we know that $\Z \times \Z$ is a group,
    so it's not empty and it's closed under the group operation.
    So $\Z \times \Z \leqslant \Q \times \Q$,
    and since $\Z \times \Z$ isn't cyclic,
    then $\Q \times \Q$ can't be either.


    \section*{Exercise 16}
    Take $|x| = n$ and $|y| = m$.
    Proof that if $x$ and $y$ commute,
    then $|xy| \mid \lcm(n, m)$: \\
    We know from a book proposition that if $x^p = 1$, $p \mid |x|$.
    We also know that if $xy$ commute,
    then by exercise 1.1.1.24, $(xy)^{|xy|} = x^{|xy|}y^{|xy|}$.
    We also know that by definition, $(xy)^{|xy|} = 1$,
    so $x^{|xy|}y^{|xy|} = 1$.
    So by the proposition we mentioned,
    $|xy| \mid |x|$ and $|xy| \mid |y|$,
    so $|xy| \mid n$ and $|xy| \mid m$.
    Since $\lcm(n, m)$ is a multiple of both $n$ and $m$ by definition,
    $|xy| \mid \lcm(n, m)$. \\
    This doesn't necessarily apply if the two don't commute,
    since our proof uses a proposition that only works if they do.
    To show it won't always apply, we will give a counterexample
    where $x$ and $y$ don't commute.
    In the symmetric group $S_3$,
    we have $|(1\;2)| = 2$ and $|(1\;2\;3\;4)| = 4$,
    which don't commute because they are not disjoint.
    We have $(1\;2) \circ (1\;2\;3\;4) = (2\;3\;4)$.
    We know that the order of a 3-tuple, $(2\;3\;4)$, is 3, 
    and $3 \nmid \lcm(2, 4) = 4$, 
    which is a counterexample. \\
    Now, to give an example where $x$ and $y$ do commute,
    and $|xy| \mid \lcm(|x|, |y|)$, but $|xy| \neq \lcm(|x|, |y|)$.
    We can consider the Quaternion group $Q_8$,
    where $|-1| = 2$ and $|i| = 4$.
    We have $|-1 \cdot i| = |-i| = 4$,
    and $|-1| \cdot |i| = 8 \neq 4$.


    \section*{Exercise 17}
    Since $Z_n$ is generated by one element,
    its presentation can contain a single generator
    so long as the relations indicates that $n$ is the first power
    to which $x$ becomes the identity.
    So, $Z_n = \langle x \mid x^n = 1 \rangle$.


    \section*{Exercise 18 $***$}
    Proof that if $H$ is a group and $h \in H$
    such that $h^n = 1$,
    then there is a unique homomorphism $\phi: Z_n \to H$
    where $Z_n = \langle x \rangle$ defined by $\phi(x) = h$: \\
    Since the function only requires that $\phi(x) = h$,
    we can add a new constraint to $\phi$ which says that 
    $\phi(x^n) = h^n$ for any $n$.
    Since the given condition that $\phi(x) = h$ still holds for $n = 1$,
    this is perfectly allowed. \\
    First, we need to show that the map is well defined.
    That means that for $x^p = x^q$, $\phi(x^p) = \phi(x^q)$.
    From exercise 1.1.1.35, and the use of the division
    algorithm, we know that for $x^n = 1$, $x^m = x^{(m \mod n)}$
    (applies whether or not $n$ is the order of $x$).
    So in $Z_n$, if $x^p = x^q$, then $p \equiv q \mod n$. \\
    We have $\phi(x^p) = h^p$ and $\phi(x^q) = h^q$.
    Since $h^n = 1$ by assumption, then by exercise 1.1.1.35
    and the proposition we wrote above,
    $h^m = x^{(m \mod n)}$ for any $m \in \Z$.
    Since we have $p \equiv q \mod n$, that must mean
    that
    \[ h^p = h^{(p \mod n)} = h^{(q \mod n)} = h^{q} \]
    So $\phi$ is well defined. \\
    Now, we can see that
    \[ \phi(x^px^q) = \phi(x^{p + q})
    = h^{p + q}
    = h^ph^q
    = \phi(x^p)\phi(x^q)  \]
    which makes $\phi$ a homomorphism. \\
    Finally, to show that the homomorphism $\phi$, with the given conditions,
    is unique,
    consider a second homomorphism $\psi: Z_n \to H$
    also with $\psi(x) = h$. \\
    Now, we added a new constraint by assuming that $\phi$ was
    defined by $\phi(x^n) = h^n$,
    which makes $\phi$ a unique homomorphism.
    But this extra constraint was not a given requirement,
    so it's possible for another homomorphism to satisfy
    the given requirements while not having the same mappings as $\phi$
    after we added our new constraint.
    So if we can show that for $\psi$, a homomorphism,
    the property $\psi(x^n) = h^n$ holds without assuming it to be the case,
    it would mean that $\psi$ is the same map $\phi$,
    and by extension that any homomorphism that has to follow the given
    requirements also satisfies the constraint we added,
    meaning they are all the same as $\phi$, which is unique. \\
    We can show that if $\psi$ is a homomorphism,
    then $\psi(x^n) = h^n$ must hold.
    We do this by induction: \\
    \textbf{Basis step:}
    We know that for $n = 2$,
    because $\psi$ is a homomorphism,
    $\psi(x^2) = \psi(xx)
    = \psi(x) \cdot \phi(x) = h \cdot h = h^2$. \\
    \textbf{Inductive hypothesis:}
    Assume that for $n = k$, $\psi(x^k) = h^k$. \\ 
    \textbf{Inductive step:}
    For $n = k + 1$, because $\psi$ is a homomorphism,
    $\psi(x^{k + 1}) = \psi(x^k + x^1)
    = \psi(x^k) \cdot \psi(x) = h^k \cdot h = h^{k+1}$. \\
    So a general function $\psi$ satisfies the added constraint,
    meaning that $\phi = \psi$,
    making $\psi$ a unique map.


    \section*{Exercise 19 $***$}
    Proof that if $H$ is a group and $h \in H$,
    then there is a unique homomorphism $\phi: \Z \to H$
    where $\Z$ is the additive group of integers
    and the $\phi$ is defined by $\phi(1) = h$: \\
    Since the function only requires that $\phi(1) = h$,
    we can add a new constraint to $\phi$ which says that 
    $\phi(n) = h^n$ for any $n \in \Z$.
    Since the given condition that $\phi(1) = h$ still holds for $n = 1$,
    this is perfectly allowed. \\
    First, we need to show that the map is well defined.
    That means that for $x, y \in \Z$, if $x = y$,
    then $\phi(x) = \phi(y)$.
    This is trivial,
    as $\phi(x) = \phi(y)$ implies that $h^y = h^x$,
    so $h^{x - y} = 1$,
    which is true since $x = y$,
    so $x - y = 0$.
    Hence $\phi$ is well defined. \\
    Now, we can see that
    \[ \phi(x + y) = h^{x + y}
    = h^xh^y
    = \phi(x)\phi(y) \]
    which makes $\phi$ a homomorphism. \\
    Finally, to show that the homomorphism $\phi$, with the given conditions,
    is unique,
    consider a second homomorphism $\psi: \Z \to H$
    also with $\psi(1) = h$. \\
    Now, we added a new constraint by assuming that $\phi$ was
    defined by $\phi(n) = h^n$,
    which makes $\phi$ a unique homomorphism.
    But this extra constraint was not a given requirement,
    so it's possible for another homomorphism to satisfy
    the given requirements while not having the same mappings as $\phi$
    after we added our new constraint.
    So if we can show that for $\psi$, a homomorphism,
    the property $\psi(n) = h^n$ holds without assuming it to be the case,
    it would mean that $\psi$ is the same map $\phi$,
    and by extension that any homomorphism that has to follow the given
    requirements also satisfies the constraint we added,
    meaning they are all the same as $\phi$, which is unique. \\
    We can show that if $\psi$ is a homomorphism,
    then $\psi(n) = h^n$ must hold.
    We do this by induction: \\
    \textbf{Basis step:}
    We know that for $n = 2$,
    because $\psi$ is a homomorphism,
    $\psi(2) = \psi(1 + 1)
    = \psi(1) \cdot \phi(1) = h^1 \cdot h^1 = h^2$. \\
    \textbf{Inductive hypothesis:}
    Assume that for $n = k$, $\psi(k) = h^k$. \\ 
    \textbf{Inductive step:}
    For $n = k + 1$, because $\psi$ is a homomorphism,
    $\psi(k + 1) = \psi(k) \cdot \psi(1)
    = h^k \cdot h^1 = h^{k+1}$. \\
    So a general function $\psi$ satisfies the added constraint,
    meaning that $\phi = \psi$,
    making $\psi$ a unique map.


    \section*{Exercise 20}
    For a prime $p$ and a positive integer $n$, proof that if $x^{p^n} = 1$,
    then $|x| = p^m$ for some positive integer $m \leqslant n$: \\
    We know that if $x^k = 1$, $|x| \mid k$.
    So $|x| \mid p^n$.
    Since $p$ is prime, the only prime factor in $p^n$'s
    prime factorization by the Fundamental Theorem of Arithmetic
    is $p$.
    So if $|x|$ divides $p^n$, then it must only contain $p$ factors.
    So $|x| = p^m$ for some positive integer $m$.


    \section*{Exercise 21 $***$}
    Let $p$ be an odd prime and $n$ a positive integer.
    Proof that $(1 + p)^{p^{n-1}} \equiv 1 \mod p^n$: \\ 
    According to the binomial theorem
    \[ (x + y)^n = \sum_{k = 0}^{n} C_{n}^{k}x^{n-k}y^k \]
    where $C_{k}^{n}$ is $n$ choose $k$. This means that 
    \[ (1 + p)^{p^{n-1}}
    = \sum_{k = 0}^{p^{n-1}} C_{p^{n-1}}^{k}1^{(p^{n-1} - k)}p^k
    = \sum_{k = 0}^{p^{n-1}} C_{p^{n-1}}^{k}p^k \]
    \[ = C_{p^{n-1}}^0p^0 + C_{p^{n-1}}^1p^1 + C_{p^{n-1}}^2p^{2}
    \dots C_{p^{n-1}}^{p^{n-1}}p^{p^{n-1}} \]
    \[ = 1 + C_{p^{n-1}}^1p^1 + C_{p^{n-1}}^2p^2
    \dots C_{p^{n-1}}^{p^{n-1}}p^{p^{-1}} \]
    \[ = 1 + \dfrac{p^{n-1}!}{1!(p^{n-1} - 1)!}p^1
    + \dfrac{p^{n-1}!}{2!(p^{n-1} - 2)!}p^2 \dots
    \dfrac{p^{n-1}!}{p^{n-1}!(p^{n-1} - p^{n-1})!}p^{p^{n-1}} \]
    where the general term looks like
    \[ \dfrac{p^{n-1}!}{k!(p^{n-1} - k)!}p^k \]
    We can factor out a single $p$, and get
    \[ \dfrac{p \cdot p^{n-1!}}{k!(p^{n-1} - k)!}p^{k-1}
    = \dfrac{p \cdot p^{n-1}(p^{n-1} - 1)!}{k!(p^{n-1} - k)!}p^{k-1}
    = \dfrac{p^n(p^{n-1} - 1)!}{(k+1)(k!)(p^{n-1} - k + 1 - 1)!}p^{k-1} \]
    \[ = \dfrac{p^n(p^{n-1} - 1)!}{(k + 1)!((p^{n-1} - 1)-(k + 1))!}p^{k-1}
    = (k+1)p^n \dfrac{(p^{n-1}-1)!}{(k + 1)!((p^{n-1}-1)-(k + 1))!}p^{k-1} \]
    \[ = (k+1)p^n C_{(p^{n-1} - 1)}^{(k+1)} p^{k-1} \]
    So we have that the sum is equal to
    \[ 1 + (1 + 1)p^nC_{p^{n-1} - 1}^{1 + 1}p^{1-1}
    + (2+1)p^nC_{p^{n-1} - 1}^{1 + 2}p^{2-1} \dots
    (p^{n-1}+1)p^nC_{p^{n-1} - 1}^{p^{n-1} + 1}p^{p^{n-1}-1} \]
    \[ = 1 + p^n(2C_{p^{n-1} - 1}^{2}p^{0}
    + 3C_{p^{n-1} - 1}^{3}p^{1} \dots
    (p^{n-1}+1)C_{p^{n-1} - 1}^{p^{n-1} + 1}p^{p^{n-1}-1}) \]
    Since each element in the sum is an integer
    (all combinations are integers),
    we can replace them by some $t \in \Z$:
    \[ 1 + p^nt\]
    This means that $(1 + p)^{p^{n-1}} - 1 = 1 + p^nt - 1
    = p^nt$,
    so $p^n \mid ((1 + p)^{p^{n-1}} - 1)$,
    which means that $(1 + p)^{p^{n-1}} \equiv 1 \mod p^n$. \\
    Let $p$ be an odd prime and $n$ a positive integer.
    Proof that $(1 + p)^{p^{n-2}} \not\equiv 1 \mod p^n$:
    We can repeat the same steps as before to get 
    \[ (1 + p)^{p^{n-2}}
    = \sum_{k = 0}^{p^{n-2}} C_{p^{n-2}}^{k}1^{(p^{n-2} - k)}p^k
    = \sum_{k = 0}^{p^{n-2}} C_{p^{n-2}}^{k}p^k \]
    \[ = C_{p^{n-2}}^0p^0 + C_{p^{n-2}}^1p^1 + C_{p^{n-2}}^2p^{2}
    \dots C_{p^{n-2}}^{p^{n-2}}p^{p^{n-2}} \]
    \[ = 1 + C_{p^{n-2}}^1p^1 + C_{p^{n-2}}^2p^2
    \dots C_{p^{n-2}}^{p^{n-2}}p^{p^{n-2}} \]
    \[ = 1 + \dfrac{p^{n-1}!}{1!(p^{n-1} - 1)!}p^1
    + \dfrac{p^{n-2}!}{2!(p^{n-2} - 2)!}p^2 \dots
    \dfrac{p^{n-2}!}{p^{n-2}!(p^{n-2} - p^{n-2})!}p^{p^{n-2}} \]
    where the general term looks like
    \[ \dfrac{p^{n-2}!}{k!(p^{n-2} - k)!}p^k \]
    However, now, we have to factor $p^2$ in order to turn
    $p^{n-2}$ into $p^n$, so we have
    \[ \dfrac{p^2 \cdot p^{n-2!}}{k!(p^{n-2} - k)!}p^{k-2}
    = \dfrac{p^2 \cdot p^{n-2}(p^{n-2} - 1)!}{k!(p^{n-2} - k)!}p^{k-2}
    = \dfrac{p^n(p^{n-2} - 1)!}{(k+1)(k!)(p^{n-2} - k + 1 - 1)!}p^{k-2} \]
    \[ = \dfrac{p^n(p^{n-2} - 1)!}{(k + 1)!((p^{n-2} - 1)-(k + 1))!}p^{k-2}
    = (k+1)p^n \dfrac{(p^{n-2}-1)!}{(k + 1)!((p^{n-2}-1)-(k + 1))!}p^{k-2} \]
    \[ = (k+1)p^n C_{(p^{n-2} - 1)}^{(k+1)} p^{k-2} \]
    So now the sum is equal to
    \[ 1 + (1 + 1)p^nC_{p^{n-2} - 1}^{1 + 1}p^{1-2}
    + (2+1)p^nC_{p^{n-2} - 1}^{1 + 2}p^{2-2} \dots
    (p^{n-2}+1)p^nC_{p^{n-2} - 1}^{p^{n-2} + 1}p^{p^{n-1}-2} \]
    \[ = 1 + p^n(2C_{p^{n-1} - 1}^{2}p^{-1}
    + 3C_{p^{n-1} - 1}^{3}p^{0} \dots
    (p^{n-2}+1)C_{p^{n-2} - 1}^{p^{n-2} + 1}p^{p^{n-2}-2}) \]
    \[ = 1 + p^n( \dfrac{2C_{p^{n-1} - 1}^{2}}{p}
    + 3C_{p^{n-1} - 1}^{3}p^{0} \dots
    (p^{n-2}+1)C_{p^{n-2} - 1}^{p^{n-2} + 1}p^{p^{n-2}-2}) \]
    where the term
    \[ \dfrac{2C_{p^{n-1} - 1}^{2}}{p}
    = \dfrac{2(p^{n-1} - 1)!}{2!(p^{n-1} - 1 - 2)!p} 
    = \dfrac{2(p^{n-1} - 1)!}{2!(p^{n-1} - 3)!p}
    = \dfrac{2(p^{n-1} - 1)(p^{n-1} - 2)}{2p} \]
    \[ = \dfrac{(p^{n-1} - 1)(p^{n-1} - 2)}{p}
    = \dfrac{p^{n-1}p^{n-1} - 3p^{n-1} + 2}{p}
    = \dfrac{p^{2n-2} - 3p^{n-1} + 2}{p} \]
    \[ = \dfrac{p^{2n-2} - 3p^{n-1}}{p} + \dfrac{2}{p} 
    = (p^{2n-3} - 3p^{n-2}) + \dfrac{2}{p} \]
    The first part of the resulting sum is an integer.
    The second part is not an integer since we assumed that
    $p$ was an odd prime, making it larger than 2
    (1 is not prime either).
    And integer plus a non-integer is a non-integer,
    so the whole term is not an integer.
    Going back to the sum, we have 
    \[1 + p^n( \underbrace{\dfrac{ 2C_{p^{n-1} - 1}^{2}}{p}}_{a}
    + \underbrace{3C_{p^{n-1} - 1}^{3}p^{0} \dots
    (p^{n-2}+1)C_{p^{n-2} - 1}^{p^{n-2} + 1}p^{p^{n-2}-2})}_{b} \]
    where $a$ is not an integer, and $b$ is an integer.
    So by the same argument, the sum is not an integer.
    So for some non-integer $u$, we can write the sum as
    \[1 + p^nu\]
    This means that $(1 + p)^{p^{n-2}} - 1 = 1 + p^nu - 1
    = p^nu$.
    Since $u$ is not an integer, $p^n \nmid ((1 + p)^{p^{n-2}} - 1)$,
    which means that $(1 + p)^{p^{n-2}} \not\equiv 1 \mod p^n$. \\
    Proof that the order of $p + 1$
    in $(\Z/p^n\Z)^\times$ is $p^{n-1}$: \\
    First, since $(\Z/p^n\Z)^\times$ is a cyclic group,
    we know from the book that if $(p + 1)^x = 1$,
    then $|p + 1| \mid x$.
    We know that $(p + 1)^{p^{n-1}} \equiv 1 \mod p^n$,
    so by definition, $(\olsi{p + 1})^{p^{n-1}} = \olsi{1}$
    in $(\Z/p^n\Z)^\times$,
    so $|p + 1| \mid p^{n-1}$.
    Since $p$ is prime, this must mean that $|p+1|$ is a power of $p$
    (only a smaller power of a prime can divide a power of the same prime).
    We also have that $|p + 1| \nmid p^{n-2}$,
    so as $|p+1|$ is a power of $p$,
    the only way it does not divide $p^{n-2}$ is if it's
    a larger power of $p$.
    So $|p + 1| = p^{n-1}$.


    \section*{Exercise 22 $***$}
    Let $n$ be an integer such that $n \geqslant 3$.
    Proof that $(1 + 2^2)^{2^{n-2}} \equiv 1 \mod 2^n$: \\ 
    We know from exercise 1.2.3.21 that for
    an odd prime $p$ and a positive integer $n$,
    the above statement applies.
    We can't use it here necessarily because in this case,
    we have a prime $p = 2$, which is even,
    so we will have to redo the proof. \\
    However, instead of using the same method,
    we first use induction to show
    that $2^n \mid ((1 + 2^2)^{2^{n-2}} - 1)$: \\
    \textbf{Basis step:} For $n = 3$,
    we have $(1 + 2^2)^{2^{n-2}} - 1 = (5)^2 - 1 = 24$,
    and $2^3 = 8$ divides $24$. \\
    \textbf{Inductive hypothesis:} For $n = k$,
    assume that $2^k \mid ((1 + 2^2)^{2^{k-2}} - 1)$. \\
    \textbf{Inductive step:} For $n = k + 1$,
    we have $(1 + 2^2)^{2^{(k+1)-2}} - 1 = (1 + 2^2)^{(2^{k-2})2} - 1
    =  ((1 + 2^2)^{(2^{k-2})})^2 - 1
    = ((1 + 2^2)^{(2^{k-2})} - 1)((1 + 2^2)^{(2^{k-2})} + 1)$.
    We assumed that $2^k \mid ((1 + 2^2)^{(2^{k-2})} - 1)$,
    and we know that $(1 + 2^2)^{(2^{k-2})}$ is a power of 5
    and is such odd, which makes $((1 + 2^2)^{(2^{k-2})} - 1)$ even,
    so $2 \mid ((1 + 2^2)^{(2^{k-2})} - 1)$.
    This means that
    $2^{k+1} \mid ((1 + 2^2)^{(2^{k-2})} - 1)((1 + 2^2)^{(2^{k-2})} + 1)$,
    which means that $2^{k+1} \mid ((1 + 2^2)^{2^{k + 1 - 2}} - 1)$. \\
    This means that $(1 + 2^2)^{2^{n-2}} \equiv 1 \mod 2^n$,
    completing the proof. \\
    Let $n$ be an integer such that $n \geqslant 3$.
    Proof that $(1 + 2^2)^{2^{n-3}} \not\equiv 1 \mod 2^n$: \\ 
    We also use induction to show that 
    that $(1 + 2^2)^{2^{n-3}} \equiv 1 + 2^{n-1} \mod 2^n$: \\
    \textbf{Basis step:} For $n = 3$,
    we have $(1 + 2^2)^{2^{n-3}} = (5)^1 = 5$,
    and $2^3 = 8$, and $1 + 2^2 = 5$.
    We know that $5 \equiv 5 \mod 8$. \\
    \textbf{Inductive hypothesis:} For $n = k$,
    assume that $(1 + 2^2)^{2^{k-3}} \equiv 1 + 2^{k-1} \mod 2^k$. \\
    \textbf{Inductive step:} For $n = k + 1$,
    we have $(1 + 2^2)^{2^{k - 3}} \equiv 1 + 2^{k-1} \mod 2^k$,
    so $(1 + 2^2)^{2^{k - 3}} - 1 - 2^{k-1} = c2^k$.
    Which means that $c2^k + 2^{k-1} + 1 = (1 + 2^2)^{2^{k - 3}}$,
    and $(c2^k + 2^{k-1} + 1)^2 = ((1 + 2^2)^{2^{k - 3}})^2$
    which implies that $(1 + 2^2)^{2^{(k+1) - 3}}
    = (c2^k + 2^{k-1} + 1)^2
    = (c2^k)^2 + 2(c2^k)(2^{k-1} + 1) + (2^{k-1} + 1)^2
    = c2^{2k} + (c2^{k+1})(2^{k-1} + 1) + (2^{k-1})^2 + 2(2^{k-1}) + 1^2
    = c2^{2k} + c2^{2k} + c2^{k+1} + 2^{2k-2} + 2^{k} + 1$ \\
    So this means that 
    $(1 + 2^2)^{2^{(k+1) - 3}} - 1 - 2^{k} = 
    = c2^{2k} + c2^{2k} + c2^{k+1} + 2^{2k-2}
    = 2^{k+1}(c2^{k - 1} + c2^{k-1} + c^2 + 2^{k-3})$
    where $(c2^{k - 1} + c2^{k-1} + c^2 + 2^{k-3})$ is an integer
    since $k \geqslant 3$.
    So $2^{k+1} \mid (1 + 2^2)^{2^{(k+1) - 3}} - (1 + 2^{k})$,
    so $(1 + 2^2)^{2^{(k+1) - 3}} \equiv 1 + 2^{k} \mod 2^{k+1}$.
    So we now know that $(1 + 2^2)^{2^{n-3}} \not\equiv 1 \mod 2^n$. \\ 
    Proof that the order of 5
    in $(\Z/2^n\Z)^\times$ is $2^{n-2}$: \\
    First, since $(\Z/2^n\Z)^\times$ is a cyclic group,
    we know from the book that if $5^x = 1$,
    then $|5| \mid x$.
    We know that $5^{2^{n-2}} \equiv 1 \mod p^n$,
    so by definition, $(\olsi{5})^{2^{n-2}} = \olsi{1}$
    in $(\Z/2^n\Z)^\times$,
    so $|5| \mid 2^{n-2}$.
    Since 2 is prime, this must mean that $|5|$ is a power of 2
    (only a smaller power of a prime can divide a power of the same prime).
    We also have that $|5| \nmid 2^{n-3}$,
    so as $|5|$ is a power of 2,
    the only way it does not divide $2^{n-3}$ is if it's
    a larger power of 2.
    So $|5| = 2^{n-2}$.


    \section*{Exercise 23 $***$}
    Proof that $(\Z/2^n\Z)^\times$ is not cyclic for $n \geqslant 3$: \\
    We know that $H = \{1, \olsi{2^{n-1} - 1}\}$
    is a subgroup of $(\Z/2^n\Z)^\times$ of order 2.
    This is because it is a subset of $(\Z/2^n\Z)^\times$,
    it's not empty, 
    and $\olsi{2^{n-1} - 1} \times \olsi{2^{n-1} - 1}
    = \olsi{2^{2n-2} - 2^n + 1} = \olsi{2^n2^{n-2} - 2^n + 1}
    =  \olsi{2^n(2^{n-2} - 1) + 1}$
    which means it is congruent to  $1 \mod 2^n$,
    so it's equal to $\olsi{1}$, the identity,
    making the set closed under inverses and mutliplication.
    So $H \leqslant (\Z/2^n\Z)^\times$. \\
    Likewise we know that $K = \{1, \olsi{2^n - 1}\}$
    is a subgroup of $(\Z/2^n\Z)^\times$ of order 2.
    This is because it is a subset of $(\Z/2^n\Z)^\times$,
    it's not empty, 
    and $\olsi{2^n - 1} \times \olsi{2^n - 1}
    = \olsi{2^{2n} - 2^{n+1} + 1} = \olsi{2^n2^n - 2^n2 + 1}
    =  \olsi{2^n(2^n - 2) + 1}$
    which means it is congruent to  $1 \mod 2^n$,
    so it's equal to $\olsi{1}$, the identity,
    making the set closed under inverses and mutliplication.
    So $K \leqslant (\Z/2^n\Z)^\times$. \\
    We know that for each cyclic group,
    there is a unique subgroup of each order
    (although it could have different generators),
    each of which corresponds to a divisor of the order of the group.
    Since we just gave an example of two distinct
    subgroups of order 2,
    this means that $(\Z/2^n\Z)^\times$ isn't.


    \section*{Exercise 24 $***$}
    For a finite gorup $G$ and $x \in G$. \\
    \begin{enumerate}[label=\textbf{\alph*.}]
        \item 
            Proof that if $g \in N_G(\langle x \rangle)$,
            then $gxg^{-1} = x^a$ for some $a \in \Z$: \\
            We know that for any element in a group with finite order,
            we can use said element to generate a cyclic group.
            Since $G$ is finite and closed,
            $x$ has finite order,
            so $x$ does generate a cyclic group.
            The normalizer of $\langle x \rangle$,
            $N_G(\langle x \rangle)$,
            is the set of elements $g$ in the cyclic group
            such that $g\langle x \rangle g^{-1} = \langle x \rangle$.
            So each element $gxg^{-1}$
            is an element in $\langle x \rangle$,
            which makes it equal to $x^a$ for $a \in \Z$.
        \item
            Proof that, conversely, if $gxg^{-1} = x^a$
            for some $a \in \Z$,
            then $g \in N_G(\langle x \rangle)$: \\
            First we note that $gx^kg^{-1} = (gxg^{-1})^k$
            for $k \in \Z^+$.
            This is because
            \[ gx^kg^{-1}
            = g\underbrace{x \cdot x \cdot x \dots x}_{k}g^{-1} 
            = g\underbrace{x \cdot 1 \cdot x \cdot 1 \cdot x 
            \dots 1 \cdot x}_{k}g^{-1}  \]
            \[ = g\underbrace{x \cdot g^{-1}g \cdot x \cdot g^{-1}g \cdot x 
            \dots g^{-1}g \cdot x}_{k}g^{-1} 
            = (gxg^{-1})^k \]
            So $gx^kg^{-1} = x^{ak}$. \\
            We can extend this property to $-k$ too.
            By exercise 1.1.1.15, we have
            \[ gx^{-k}g^{-1} = g(x^{-1})^kg^{-1}
            =  (gx^{-1}g^{-1})^k
            = (((g^{-1})^{-1}(x^{-1})^{-1}g^{-1})^{-1})^k \]
            \[ = (gx^{-1}g^{-1})^{-k} \]
            So this property extends to all negative powers too.
            We also know that $gx^0g^{-1} = gg^{-1} = 1 = (gxg^{-1})^0$,
            so it applies for all $k \in \Z$. \\
            So we have $gx^kg^{-1} = (gxg^{-1})^k = (x^a)^k = x^{ak}$,
            This means that any $gx^kg^{-1}$ is in $\langle x \rangle$.
            So $g\langle x \rangle g^{-1} \subseteq \langle x \rangle$. \\
            We know from exercises 1.1.7.17
            that a set $G$ acts on a subset $A$,
            by conjugation.
            Since group actions permute the set they act on,
            that means that for $\langle x \rangle \leqslant G$,
            and $G$ acts on $\langle x \rangle$ by conjugation,
            which means that
            $|g\langle x \rangle g^{-1}| = |\langle x \rangle|$.
            So as $g\langle x \rangle g^{-1} \subseteq \langle x \rangle$,
            we have $g\langle x \rangle g^{-1} = \langle x \rangle$.
            So by definition, $g \in N_G(\langle x \rangle)$.
    \end{enumerate}


    \section*{Exercise 25}
    Let $G$ be a cyclic group of order $n$,
    and let $k$ be a integer relatively prime to $n$.
    Proof that $\psi: G \to G$ defined by $\psi(x) = x^k$ is surjective: \\
    If $x^p, x^q \in G$,
    and we assume that $0 \leqslant p, q < n$,
    then $\psi(x^p) = (x^p)^k = x^{pk}$
    and $\psi(x^p) = (x^q)^k = x^{qk}$.
    We know from exercise 1.1.1.35 that for an element $x$ of order $n$,
    $x^m = x^{(m \mod n)}$.
    Assume now that $x^{pk} = x^{qk}$.
    This means that $pk \equiv qk \mod n$,
    so $n \mid k(p - q)$.
    We assumed that $\gcd(k, n) = 0$,
    so no factors of $n$ are in $k$,
    which means that $n \mid (p - q)$.
    However, we have $0 \leqslant p, q < n$, 
    so $-n < p - q < n$.
    This means that the only way $n$ divides $(p - q)$
    is if $p - q = 0$.
    So if $\psi(x^p) = \psi(x^q)$,
    then $p = q$, 
    so $x^p = x^q$.
    This makes $\psi$ injective.
    Since $\psi$ is injective,
    and $|G| = |G|$ (since the map maps $G$ to itself),
    that must make $\psi$ surjective as well
    (since each element maps to only one image,
    and there are as many elements in the input and output,
    then each element in the output must be an image of some element).


    \section*{Exercise 26 $***$}
    Let $Z_n$ be a cyclic group of order $n$, and, for $a \in \Z$,
    let $\sigma_a: Z_n \to Z_n$ be a map
    defined by $\sigma_a(x) = x^a$ for each $x \in Z_n$. \\
    \begin{enumerate}[label=\textbf{\alph*.}]
        \item 
            Proof that $\sigma_a$ is an automorphism of $Z_n$
            if and only if $\gcd(a, n) = 1$: \\
            If $\sigma_a$ is an automorphism,
            then the map is a bijection.
            So it is an injection.
            This means that $\sigma_a(x^p) \neq \sigma_a(x^q)$
            when $x^p \neq x^q$.
            We know from exercise 1.1.1.35 that for an element $x$ of
            order $n$, $x^m = x^{(m \mod n)}$.
            So if $x^p \neq x^q$, then $p \not\equiv q \mod n$,
            so $n \nmid (p - q)$.
            Now, if $\sigma_a(x^p) \neq \sigma_a(x^q)$,
            then $x^{ap} \neq x^{aq}$,
            so by the same argument, $ap \not\equiv aq \mod n$,
            and $n \nmid a(p - q)$.
            We know that $n \nmid (p-q)$.
            However, we can't fix the values of $p$ and $q$,
            so the only way to ensure that $a$ not divide $a(p - q)$
            is for $a$ to not share any factors with $n$.
            So $a$ and $n$ are relatively prime. \\
            Conversely, if $\gcd(a, n) = 1$,
            then consider the map $\sigma_a$.
            First, we need to show that the map is well defined.
            Assume that $x^p = x^q$,
            then $\sigma_a(x^p) = (x^p)^a = (x^q)^a = \sigma_a(x^q)$.
            So $\sigma_a$ is well defined. \\
            Next, we show it's a homomorphism:
            \[ \sigma_a(x^px^q) = \sigma_a(x^{p + q})
            = (x^{p + q})^a
            = x^{ap + aq}
            = x^{ap}x^{aq}
            = \sigma_a(x^p)\sigma_a(x^q) \]
            Finally, we need to show that the map is surjective,
            which we know to be the case from
            exercise 1.2.3.25 (since $\gcd(a, n) = 1$).
            Since $\sigma_a$ maps $Z_n$ to itself,
            the input set and output have the same order,
            so the surjection becomes a bijection.
            We conclude that $\sigma_a$ is an isomorphism,
            and in this case, an automorphism.
        \item
            Proof that $\sigma_a = \sigma_b$
            if and only if $a \equiv b \mod n$: \\
            If $\sigma_a = \sigma_b$,
            then for $x \in Z_n$,
            $\sigma_a(x) = x^a = x^b = \sigma_b(x)$.
            We know from exercise 1.1.1.35 that for an element $x$ of
            order $n$, $x^m = x^{(m \mod n)}$.
            So if we assume that $Z_n = \langle x \rangle$,
            then $|x| = |Z_n| = n$,
            so $x^a = x^b$
            implies that $a \equiv b \mod n$. \\
            Conversely if $a \equiv b \mod n$,            
            then consider a random element $x^p \in Z_n$.
            We have $a \equiv b \mod n$,
            which means that $n \mid (a - b)$,
            so $a - b = kn$ for some $k \in \Z$.
            So $\sigma_a(x^p) = x^{ap}
            = x^{(kn + b)p}
            = x^{knp + bp}
            = x^{knp}x^{bp}
            = (x^n)^{kp}x^{bp}
            = 1^{kp}x^{bp}
            = x^{bp}
            = \sigma_b(x)$.
            Since it applies for all elements in $Z_n$,
            we conclude that $\sigma_a = \sigma_b$.
        \item
            Proof that every automorphism of $Z_n$ is equal
            to $\sigma_a$ for some $a \in \Z$: \\
            Assume that $\varphi:Z_n \to Z_n$ is an automorphism.
            Since it's mapping $Z_n$ to itself,
            the mapping of an element $x^k$ is $x^m$
            (since the group is cyclic).
            Then for some $k \in \Z$,
            by exercise 1.1.6.1,
            $\varphi(x^k) = \varphi(x)^k = (x^m)^k = (x^k)^m$.
            This applies to any element $x \in Z_n$,
            so $\varphi = \sigma_m$ for some $m \in \Z$. 
        \item
            Proof that $\sigma_a \circ \sigma_b = \sigma_{ab}$: \\
            We have $\sigma_a(\sigma_b(x^k))
            = \sigma_a((x^{bk}))
            = x^{abk}
            = \sigma_{ab}(x^k)$
            which proves the statement. \\
            The automorphism group of a group $G$ called $Aut(Z_n)$
            is the set of isomorphisms from $G$ to itself $G$.
            Proof that $\psi: (\Z/n\Z)^\times \to Aut(Z_n)$
            defined by $\psi{\olsi{a}} = \sigma_a$ is an isomorphism: \\
            First, we know that the map is well defined
            because for $\olsi{a} = \olsi{b}$,
            then by definition $a \equiv b \mod n$,
            so by our proof in part b,
            $\psi(\olsi{a}) = \sigma_a = \sigma_b = \varphi(\olsi{b})$. \\
            And, by the proof in part d, we have
            \[ \psi(\olsi{a} + \olsi{b})
            = \psi(\olsi{a + b})
            = \sigma_{ab}
            = \sigma_{a}\sigma_{b} 
            = \psi(\olsi{a})\psi(\olsi{b}) \]
            which means $\psi$ is a homomorphism. \\
            We also know that the map is injective.
            since by part b, $\sigma_a = \sigma_b$
            if and only if $a \equiv b \mod n$.
            So for $\olsi{a} \neq \olsi{b}$,
            by definition $a \not\equiv b \mod n$,
            so $\sigma_a \neq \sigma_b$,
            which means that $\psi(\olsi{a}) \neq \psi(\olsi{a})$,
            making $\psi$ an injection. \\
            Finally, we know from part c that any automorphism
            from $Z_n$ to itself is of the form $\sigma_a$
            for some $a \in \Z$.
            By part b $\sigma_a = \sigma_b$
            if and only if $a \equiv b \mod n$,
            so by definition we have as many
            unique automorphisms as we do unique residue classes
            modulo $n$.
            So $|(\Z/n\Z)^\times| = |Aut(Z_n)|$,
            which turns the injection into a bijection.
            We thus conclude that $\psi$ is an isomorphism,
            so $(\Z/n\Z)^\times \cong Aut(Z_n)$.
    \end{enumerate}

\end{document}
