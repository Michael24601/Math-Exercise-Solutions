

\documentclass[12pt]{article}

\usepackage[margin=1in]{geometry}

% For using float option H that places figures exatcly where we want them
\usepackage{float}
% makes figure font bold
\usepackage{caption}
\captionsetup[figure]{labelfont=bf}
% For text generation
\usepackage{lipsum}
% For drawing
\usepackage{tikz}
% For smaller or equal sign and not divide sign
\usepackage{amssymb}
% For the diagonal fraction
\usepackage{xfrac}
% For enumerating exercise parts with letters instead of numbers
\usepackage{enumitem}
% For dfrac, which forces the fraction to be in display mode (large) e
% even in math mode (small)
\usepackage{amsmath}
% For degree sign
\usepackage{gensymb}
% For "\mathbb" macro
\usepackage{amsfonts}
\newcommand{\N}{\mathbb{N}}
\newcommand{\Z}{\mathbb{Z}}
\newcommand{\Q}{\mathbb{Q}}
\newcommand{\R}{\mathbb{R}}
\newcommand{\C}{\mathbb{C}}

% overline short italic
\newcommand{\olsi}[1]{\,\overline{\!{#1}}}

\usepackage{indentfirst}
\usetikzlibrary{shapes,positioning,fit,calc}

\usepackage{changepage} % for adjustwidth environment

\title{%
    \Huge Abstract Algebra \\
    \large by \\
    \Large Dummit and Foote \\~\\
    \huge Part 1: Group Theory \\
    \LARGE Chapter 2: Subgroups \\
    \Large Section 1: Definition and Examples 
}
\date{2024-03-20}
\author{Michael Saba}

\begin{document}
    \pagenumbering{gobble}
    \maketitle
    \newpage
    \pagenumbering{arabic}

    
    We can better understand structures of mathematical objects
    by studying subsets of them that satisfy some axioms.
    Another method is studying quotients of an objects;
    a quotient group is a way of collaping a group on a smaller group,
    is another such way.
    This section (and chapter) will focus on the former.
    Chapter 1.3 will focus on the latter. \\
    
    If $G$ is a group, then $H$ is a \textbf{subgroup} of $G$
    if it a non-empty subset of $G$, and it is closed under
    inverses and the group operation
    (so for all $x, y \in H$, $xy$ must be in $H$,
    as well as $x^{-1}$ and $y^{-1}$). \\
    Another way of saying this is that $H$ is a subgroup of $G$
    if it is a subset of $G$ and a group in its own right,
    using the same binary operation as the one used in $G$
    (this immediatly gives us that the operation of $H$
    must be associative, as it is associative in $G$). \\
    For a subgroup $H$ of $G$, we write $H \leqslant G$. \\

    The group operation of a subgroup $H$ of $G$ is the same as $G$'s,
    meaning that inverses of elements in $G$
    are the same as the inverses of those same elements in $H$.
    By extension, the identity of $G$ is the identity of $H$. \\
    We know for sure that $H$ will contain the identity of $G$,
    since $H$ is closed under inverses and the group operation
    and $H$ is non-empty;
    the existence of any element $x \in H$ implies
    the existence of $x^{-1} \in H$.
    Closure under the group operation tells us that $xx^{-1} \in H$,
    so $1 \in H$. \\

    It is possible for $H$ to be a subset of $G$
    and a group in its own right using a different operation than $G$'s,
    but then it wouldn't necessarily be a subgroup of $G$.
    When we say $H \leqslant G$,
    we always mean that the binary operation of $H$ is the same as $G$'s
    restricted to $H$. \\

    Any group $G$ is a subgroup of itself,
    since $G$ is its own subset
    (the rest of the axioms are trivial to prove). \\
    If a subgroup $H$ of $G$ is a proper subset of $G$
    (e.g. is not $G$),
    then we denote it using $H < G$. \\

    When we know $G$ is a group,
    checking that any non-empty subset $H$ is as simple as making sure
    $H$ is closed under $G$'s binary operation and inverses.
    However, these two conditions can be amalgamated into a single
    condition.
    In short, $H \subseteq G$ is a subgroup of $G$
    if and only if
    \begin{itemize}[label=$\diamond$]
        \item
            $H \neq \emptyset$.
        \item
            For all $x, y \in H$, $xy^{-1} \in H$
            (which gives us closure under the group operation and inverses).
    \end{itemize}
    On one hand,
    if $H \leqslant G$, then it is trivial that these equations hold.
    On the other hand, if $H \neq \emptyset$
    and if for all $x, y \in H$, $xy^{-1} \in H$,
    then we can conclude that $1 \in H$;
    we know some $h \in H$ since it is non-empty,
    so if we take $x = h$ and $y = h$,
    then $hh^{-1} \in H$,
    so $H$ contains the identity.
    Now, if we take $x = 1$ and $y = h$ for any $h \in H$,
    we can show that $1h^{-1} \in H$,
    making it closed under inverses.
    Furthermore, now that we know we can take the inverse of any element,
    for any $g, h \in H$, we can take $x = g$ and $y = h^{-1}$,
    and we'll know that $g(h^{-1})^{-1} = gh$ is in $H$,
    making it closed under teh group operations. \\
    
    If the subset $H$ of $G$ is finite
    ($G$ need not be)
    then it is enough to show that $H$ is not-empty,
    and that $H$ is closed under the group operation,
    in order to show that it is a subgroup.
    For any element $x \in H$,
    there are finitely many distinct elements
    $x^1, x^2, x^3 \dots$ in $H$. \\
    We can continue multipying $x$ by itself beyond that,
    but then we will have to have some $x^a = x^b$
    for some positive integers $a < b$.
    So if $n = b-a$,
    then $x^n = 1$ by cancellation laws,
    such that $x^{n-1} = x^{-1}$. \\
    Since it's an element of $H$,
    it must be that $H$ is closed under inverses. \\
    

\end{document}
