
    
\documentclass[12pt]{article}

\usepackage[margin=1in]{geometry}

% For using float option H that places figures exatcly where we want them
\usepackage{float}
\usepackage{tcolorbox} % for framing
% makes figure font bold
\usepackage{caption}
\captionsetup[figure]{labelfont=bf}
% For text generation
\usepackage{lipsum}
% For drawing
\usepackage{tikz}
% For smaller or equal sign and not divide sign
\usepackage{amssymb}
% For the diagonal fraction
\usepackage{xfrac}
% For enumerating exercise parts with letters instead of numbers
\usepackage{enumitem}
% For dfrac, which forces the fraction to be in display mode (large) e
% even in math mode (small)
\usepackage{amsmath}
% For degree sign
\usepackage{gensymb}
% For "\mathbb" macro
\usepackage{amsfonts}
\newcommand{\N}{\mathbb{N}}
\newcommand{\Z}{\mathbb{Z}}
\newcommand{\Q}{\mathbb{Q}}
\newcommand{\R}{\mathbb{R}}
\newcommand{\C}{\mathbb{C}}

% overline short italic
\newcommand{\olsi}[1]{\,\overline{\!{#1}}}

\usepackage{indentfirst}
\usetikzlibrary{shapes,positioning,fit,calc}
\usetikzlibrary{arrows}

\usepackage{changepage} % for adjustwidth environment

\title{%
    \Huge Abstract Algebra \\
    \large by \\
    \Large Dummit and Foote \\~\\
    \huge Part 1: Group Theory \\
    \LARGE Chapter 1: Introduction to Groups \\
    \Large Section 4: Matrix Groups
}
\date{2024-03-11}
\author{Michael Saba}

\begin{document}
    \pagenumbering{gobble}
    \maketitle
    \newpage
    \pagenumbering{arabic}


    In the section, the notion of groups made up matrices
    will be introduced. \\
    The concept of fields will also have to be explained birefly. \\

    \subsection*{Fields}

    A \textbf{field} is the smallest mathematical structure
    in which we cn perform
    all of the arithmetic operations $+, -, \times, \div$ 
    (division can only be done by non-zero elements). \\
    A field contains both an additive and multiplicative identity,
    and all elements have both an additive and multiplicative inverse
    except for the additive identity,
    which only has an additive inverse
    (the zero element). \\
    Examples of fields include the field of rationals $\Q$
    which is infinite,
    as well as $\Z/p\Z$ where $p$ is prime,
    which is finite.
    The latter field we denote as $\mathbb{F}_p$. \\

    In order to summarize the afore mentioned properties of fields,
    we can define it in terms of groups:
    \begin{itemize}[label=$\diamond$]
        \item 
            A field is a set $F$
            along with two binary operations $+$ and $\cdot$,
            such that $(F, +)$ is an abelian group
            (whose identity we will call $0$)
            and $(F - \{0\}, \cdot)$ is also an abelian group
            (whose identity we will call $1$).
            The fact that $0$ is not part of the multiplicative group
            exempts it from having to have a multiplicative inverse,
            which corresponds to dividing by $0$. \\
            The following distributive law must also hold: 
            \[ a \cdot (b + c) = (a \cdot b) + (a \cdot c)
            \qquad \forall a, b, c \in F \]
        \item 
            For any field $F$,
            we will use $F^+$ to denote the additive group that forms it,
            and $F^\times$ to denote the multiplicative group $F - \{0\}$
            that also forms it.
    \end{itemize}

    The order of a field $F$, like that of a group,
    is size of the set $F$, denoted by $|F|$. \\

    \subsection*{Matrix Groups}

    Whenever we are usually working with matrices,
    we implicitely use the elements of the field of real numbers $\R$
    for the entries of the matrix,
    the scalars,
    and the determinants of the matrices (and vectors). \\
    However, for an arbitrary field $F$,
    all of the rules we know to be true in Linear Algebra remain true;
    we don't need to be working with reals
    in vector space and matrix theory,
    we only need a structure in which we can
    divide, add, multiply, and subtract,
    which can be any field. \\

    Now, we can define a new family of groups.
    For any fixed field $F$,
    and any $n \in \Z^+$,
    we will define
    \textbf{the general linear group of degree $\boldsymbol{n}$}
    denoted by $GL_n(F)$
    to be the set of all invertible $n \times n$ matrices
    whose entries come from $F$.
    \[ GL_n(F) = \{ A \mid A
    \text{ is an $n \times n$ matrix
    with entries from $F$ and det$(A) \neq 0$} \} \]
    The determinants can be computed with the same formula
    we use for matrices in $\R$
    (that's why we needed a field, doing operations in matrices
    requires all the operations of a field). \\

    We now prove this is a group, with matrix multiplication
    as the binary operator:
    \begin{itemize}[label=$\diamond$]
        \item 
            We know that the set of $n \times n$ invertible matrices
            is closed under multiplication.
            On one hand, the product of any two $n \times n$ matrices
            $A$ and $B$ is an $n \times n$ matrix $AB$.
            On the other hand, $\det(AB) = \det(A) \cdot \det(B)$.
            Since $A$ and $B$ were assumed to be invertible,
            neither of their determinants is $0$,
            so $\det(AB) \neq 0$,
            meaning $AB$ is also invertible. 
        \item 
            Matrix multiplication is by definition associative.
        \item 
            The identity of the group
            is the $n \times n$ identity matrix $I_n$,
            since $I_nA = AI_n = A$.
        \item 
            The inverse of a matrix $A \in GL_n(F)$
            is just the inverse of the matrix $A^{-1}$,
            which we know exists as $A$ is assumed to be invertible.
            By definition,
            $AA^{-1} = A^{-1}A = I_n$.
    \end{itemize}
    Thus, $GL_(F)$ is a group under matrix multiplication. \\

    THough these properties won't be proven until Part III,
    we will record them here for convenience:
    \begin{itemize}[label=$\diamond$]
        \item 
            If $F$ is a field and $|F| < \infty$,
            then $|F| = p^m$ for some prime $p$
            and some integer $m$.
        \item 
            If $|F| = q$, then
            $|GL_n(F)| = (q^n - 1)(q^n - q)(q^n - q^2)\dots(q^n-q^{n-1})$.
    \end{itemize}
    
\end{document}